\section{Context}

Context is information that is computationally accessible at runtime,  and describes both the execution environment (external context) and the system state (internal context).

In self-adaptive systems, the capacity of a system sense and model its context of execution is called context-awareness (for external context) and self-awareness (for internal context).

\section{Self-adaptability}

Self-adaptative have been accepeted as a promissing approach to tackle run-time
complexty and context change. Self-adaptivesses is an approach in which the system
\textit{"evaluates its own behavior and changes behavior when the evaluation indicates that it is not accomplishing what the software is intended to do, or when better functionality or performance is possible."}\cite{laddaga1997}.



% laddaga1997: it should relies on software informed about its mission and about its construction and behavior.  This implies that the software has multiple ways of accomplishing its purpose, and has enough knowledge of its construction to make effective changes at runtime.

% Such software should include functionality for evaluating its behavior and performance, and the ability to replan and reconfigure its operations in order to improve its operation.  Self adaptive software should also include a set of components for each major function, along with descriptions of the components, so that components of systems can be selected and scheduled at runtime, in response to the evaluators.

% It also requires the ability to impedance match input/output of sequenced components, and the ability to generate some of this code from specifications. In addition, we seek this new basis of adaptation to be applied at runtime, as opposed to development/design time, or as a maintenance activity.


% mape-k

% different approaches ref salehie

% challenges
