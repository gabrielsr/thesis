%!TEX root = /Users/gabrielsr/Dropbox/Projeto Final/monografia/monografia.tex
\chapter{Introdução}

Nesse capítulo introduzimos os conceitos 

\section{Engenharia de Software e Métodos Ágeis}


O termo Engenharia de Software surgiu por volta de 1968, junto com os circuitos integrados e a consequente demanda por  sistemas de software cada vez mais complexos \cite{Sommerville2007}, capazes de aproveitar as novas capacidades de hardware  na solução de um conjunto mais abrangente de problemas. Com a crescente complexidade do software necessários surgiu a necessidade da aplicação de processos, pois o desenvolvimento \emph{ad hoc} (sem que haja um processo definido) se tornava muito dispendioso.


\subsection{Processo de Software}
	Processo de Software é o roteiro de atividades que se segue objetivando produzir um software de alta qualidade em tempo. O processo organiza o desenvolvimento o afastando do caos. 
	
	Pressman \cite{Pressman2006} afirma que :
“Os processos de software formam a base para o controle gerencial de projetos de software e estabelecem o contexto no qual os métodos técnicos são aplicados, os produtos de trabalho (modelos, documentos, dados, relatórios, formulários etc.) são produzidos, os marcos são estabelecidos, a qualidade é assegurada e as modificações são adequadamente geridas". 
	
As principais atividades de um processo de software segundo Sommerville \cite{Sommerville2007} são: 
\begin{description}
\item[Especificação] É definido por clientes e engenheiros o que precisa ser desenvolvido e suas diversas restrições
\item[Desenvolvimento] O software é projetado e programado
\item[Validação] Nessa etapa o software é verificado para ver se atende ao que o cliente queria
\item[Evolução de Software] O software é modificado para se adaptar às mudanças  no requerido pelo cliente e pelo mercado
\end{description}

\subsection{Metodologias de Desenvolvimento de Software}

A prática do desenvolvimento de software é uma atividade caótica em sua maior parte, normalmente caracterizada pela expressão "codificar e consertar" \cite{MartinFowler2005}. O software é escrito sem um plano definido e o projeto do sistema é repleto de várias decisões de curto prazo. Quando o sistema é pequeno isso funciona bem - mas à medida que o sistema cresce, torna-se cada vez mais difícil adicionar novos recursos a ele. Defeitos subsequentes se tornam cada vez mais dominantes e cada vez mais difíceis de serem eliminados. Um sinal típico de um sistema desse tipo é uma longa fase de testes depois que o sistema está "pronto". Esta longa fase de testes entra em confronto direto com o cronograma, pois testes e depuração são atividades cujos tempos são dificilmente estimados com precisão.

Este estilo de desenvolvimento é utilizado há muito tempo, mas também existe uma alternativa há muito tempo: Metodologia. Metodologias impõem um processo disciplinado no desenvolvimento de software, com o objetivo de torná-lo mais previsível e mais eficiente. Elas fazem isso desenvolvendo um processo detalhado com uma forte ênfase em planejamento e inspirado em outras disciplinas de engenharia. Por isso são chamadas de \textbf{metodologias de engenharia} \cite{MartinFowler2005}.

Metodologias de engenharia estão disponíveis há muito tempo. Elas não têm sido percebidas como sendo particularmente bem-sucedidas. Elas têm sido notadas menos ainda por serem populares. A crítica mais frequente é que estas metodologias são burocráticas. Há tanta coisa a se fazer para seguir a metodologia que todo o ritmo de desenvolvimento fica mais lento.

\subsection{Metodologias Ágeis}

Um novo grupo de metodologias surgiu nos últimos anos como uma reação às metodologias tradicionais. Durante algum tempo elas foram conhecidas como metodologias leves, mas agora o termo mais aceito é \textbf{metodologia ágil}. Para muitas pessoas o apelo das metodologias ágeis é a reação delas à burocracia das metodologias tradicionais. Estas novas metodologias tentam criar um equilíbrio entre nenhum processo e muito processo, provendo apenas o suficiente de processo para obter um retorno razoável.

O resultado disso tudo é que os métodos ágeis têm algumas mudanças de ênfase significativas em relação aos métodos de engenharia. A diferença mais evidente é que metodologias ágeis são menos centradas em documentação, normalmente enfatizando uma quantidade menor de documentos para uma dada tarefa. De várias maneiras, elas são mais voltadas ao código-fonte do programa: seguindo um caminho que diz que a parte-chave da documentação é o próprio código-fonte.

Menos documentação é apenas um sintoma de duas diferenças mais profundas, segundo Martin Fowler \cite{MartinFowler2005}:

\begin{description}
\item[Metodologias ágeis são adaptativas ao invés de predeterminantes.] Metodologias de engenharia tendem a tentar planejar uma grande parte do processo de desenvolvimento detalhadamente por um longo período de tempo. Isso funciona bem até as coisas mudarem. Então a natureza de tais métodos é a de resistir à mudança. Para os métodos ágeis, entretanto, mudanças são bem-vindas. Eles tentam ser processos que se adaptam e se fortalecem com as mudanças, até mesmo ao ponto de se auto-modificarem.

\item[Métodos ágeis são orientados a pessoas ao invés de serem orientados a processos.] O objetivo dos métodos de engenharia é de definir um processo que irá funcionar bem, independentemente de quem os estiverem utilizando. Métodos ágeis afirmam que nenhum processo jamais será equivalente à habilidade da equipe de desenvolvimento. Portanto, o papel do processo é dar suporte à equipe de desenvolvimento e seu trabalho.
\end{description}

Processos ágeis têm por objetivo criar software útil o mais rápido possível e responder a mudanças de requisitos com o menor custo. Apesar de existirem diferentes abordagens e não existir uma definição clara para processo ágil é possível identificar características comuns em uma família de processos chamados ágeis. Estes processos são iterativos, geram pouca documentação, o software é entregue incrementalmente e desenvolvido em ciclos de curta duração.

\section{Processos Ágeis}

	O marco inicial do “Movimento Ágil” na indústria de software é a publicação do “Manifesto Ágil” em 2001\cite{agilemanifesto2001}. Esse manifesto apresenta os seguintes pontos chaves na sua introdução:

	“Estamos descobrindo maneiras melhores de desenvolver software fazendo-o nós mesmos e ajudando outros a fazê-lo. Através desse trabalho, passamos a valorizar:

\begin{itemize}
\item Indivíduos e interação entre eles mais que processos e ferramentas;

\item Software em funcionamento mais que documentação abrangente;

\item Colaboração com o cliente mais que negociação de contratos;

\item Responder a mudanças mais que seguir um plano.
\end{itemize}
Ou seja, mesmo havendo valor nos itens à direita, valorizamos mais os itens à esquerda.” \\

Esse documento é assinado com um manifesto em alusão a movimentos de vanguarda na arte e na política. É bem o que o movimento ágil pretendia ser, quebrando os paradigmas da Engenharia de Software vigente, introduzindo uma nova forma de pensar em Engenharia, bem distinto dos processos sólidos e burocráticos então vigentes.

Os processos ágeis privilegiam a entrega de software útil ao cliente ao invés de produção de documentos que não agregam valor ao software. Nos processos ágeis não é feita uma análise de requisitos detalhada no início do processo, ao invés disso são detalhados alguns requisitos iniciais, suficientes para se iniciar o desenvolvimento.

Cada versão do software é desenvolvida para atender a um novo e pequeno conjunto de requisitos. Essa nova versão é entregue ao cliente, que faz críticas e junto ao desenvolvedor levanta novos requisitos. Assim o software é desenvolvido incrementalmente a medida que os requisitos vão se tornando mais claros. 

Processos iterativos segundo Sommerville \cite{Sommerville2007} são adequados para desenvolvimento de software para negócios, de \emph{e-commerce} e pessoais, pois refletem a maneira fundamental como as pessoas tendem a pensar nos problemas. Raramente  as pessoas trabalham na solução completa antecipadamente, mas seguem em direção à solução por meio de uma série de passos, retrocedendo quando percebem que cometeram um erro.  O desenvolvimento iterativo é preferível porque nesses casos é quase impossível entender todo o problema antes que uma parte do sistema esteja em operação. 

No entanto os processos ágeis têm algumas limitações e não são aplicados a todos os sistemas de software. Os principais problemas citados por Sommerville \cite{Sommerville2007} com desenvolvimento iterativo são:	

\begin{description}
\item[É difícil estimar custo com antecedência.] Como os requisitos do sistema só são descobertos a medida que é desenvolvido não podemos no início do progresso estimar os custos de desenvolvimento. Uma organização que produza software para si mesma terá dificuldades em avaliar quanto de recursos será destinado ao projeto e se é  mais lucrativo desenvolvê-lo ou comprar uma solução já pronta. Uma empresa que desenvolva software sob encomenda terá dificuldade em fazer um orçamento.

\item[É difícil de acompanhar o desenvolvimento do projeto.] Em um processo de desenvolvimento ágil não sabemos inicialmente a complexidade do projeto desenvolvido, todas as funções a serem implementadas e consequentemente não temos um cronograma de atividades. Isso dificulta a gerência acompanhar o progresso do projeto e prever quando estará concluído. A falta de documentação dificulta rastrear a introdução de erros e coletar dados que possibilitem uma melhora do processo.

\item[Desaconselhável para projetos que envolvam alto risco e alto custo de teste.] \newline

Processos ágeis por sua natureza iterativa são baseados em uma boa carga de testes. Em um projeto que seja muito caro ou impossível de se testar o sistema, como no caso de software embarcado em uma aeronave,  é muito ineficiente utilizar um método ágil.  Os métodos ágeis adequados a áreas que seja mais importante entregar um software rápido ao usuário do que entregar um sistema plenamente confiável.
\end{description}

\section{O SCRUM}

O SCRUM está entre os métodos ágeis mais utilizados hoje. Seu criador, Ken Schwaber foi um dos signatários originais do Manifesto Ágil \cite{agilemanifesto2001}. O SCRUM é baseado em práticas aceitas pelo mercado, utilizadas por décadas. Ele é definido então em uma teoria de processos empíricos.

É uma metodologia de desenvolvimento de produto consistindo de práticas e regras utilizadas pela administração, pelos clientes e pela gerência de projeto para maximizar a produtividade e o valor do esforço de desenvolvimento. Ele não é um processo ou uma técnica para o desenvolvimento de produtos. Ao invés disso, é um \emph{framework} dentro do qual você pode empregar diversos processos e técnicas  \cite{Schwaber2002}. Neste trabalho ele será referenciado como \emph{framework} SCRUM.

Tem como principais características ágeis a adaptatividade e o desenvolvimento com foco nas pessoas e nas interações entre elas.

\section{Modelo de Qualidade de Processo}

Existem hoje várias abordagens diferentes para desenvolvimento de software, não existindo abordagem ideal. Dado esse contexto surge a importância de modelos de qualidade de processo, através do qual se pode avaliar organizações quanto a sua maturidade em relação ao desenvolvimento de software independente da abordagem utilizada.

Um modelo de qualidade de processo descreve boas práticas de Engenharia de Software que um processo deve evidenciar para ter um certo nível de maturidade. O CMMI (\emph{Capability Maturity Model Integration}) é um modelo de qualidade de processo mundialmente reconhecido. O MPS.BR ou Melhoria de Processos do Software Brasileiro é simultaneamente um movimento para a melhoria da qualidade (Programa MPS.BR) e um modelo de qualidade de processo (Modelo MPS).O Modelo MPS.BR é compatível com CMMI e voltado para a realidade brasileira, sobre tudo para as pequenas e médias empresas (PMEs).

As organizações responsáveis pelos modelos como CMMI e MPS.BR são também responsáveis por modelos de avaliação e programa de certificação que avaliam e atestam a maturidade do processo.

\section{Motivação}


\section{Projeto}
\subsection{Justificativa}
\subsection{Objetivo Geral}	

\subsection{Objetivos específicos}
\begin{itemize}
 \item Definir o processo
  \item Implantar o processo;
  \item Fazer ..
  \item Avaliar ...
\end{itemize}
	
\subsection{Resultados Esperados}
Esperamos 
\subsection{Estrutura do Trabalho} 

O presente trabalho está dividido em 5 capítulos ...
O Capítulo 2 ...