\documentclass[mestrado]{pacotes/unb-cic}
\usepackage[american,brazil]{babel}
\usepackage[T1]{fontenc}
\usepackage{indentfirst}
\usepackage{natbib}
\usepackage{xcolor,graphicx,url}
\usepackage[utf8]{inputenc}
\usepackage{booktabs} % for tables
\usepackage{setspace}%%% more space for advisor notes
\usepackage{verbatim} %% multiline comments
\usepackage{dirtytalk} %% Quotes
\usepackage{pacotes/tikz-uml} % UML
\usepackage[autostyle]{csquotes}

\graphicspath{ {imagens/} } % path to images

%\bibpunct[; ]{(}{)}{,}{a}{}{;}%muda colchetes para parenteses

% definicoes previas do documento
%\selectlanguage{brazil}
%\title{A Runtime model and middleware for requirements-, structure- and context-aware systems }
\title{A Goal-Oriented Middleware for Dependable Self-Adaptive Systems}

\orientador[a]{\prof[a] \dr[a] Genaina Nunes Rodrigues}{CIC/UnB}

\coordenador[a]{\prof[a] \dr[a] Alba Cristina Magalhaes Alves de Melo}{CIC/UnB}

\diamesano{9}{Junho}{2016}


\membrobanca{\prof }{CIC/UnB}

\membrobanca{\prof }{CIC/UnB}

\autor{Gabriel Siqueira}{Rodrigues}

\CDU{004.4}

\palavraschave{dependabilidade}
\keywords{dependability}

%-------------------------------------------------

\begin{document}

\maketitle

\pretextual
%\begin{agradecimentos}
%Agradecemos à nossa orientadora, \prof[a] \dr[a] Genaina Nunes Rodrigues,
%\end{agradecimentos}

%\begin{resumo}
%Nesse trabalho apresentamos...
%\end{resumo}

\selectlanguage{american}

\begin{abstract}\textbf{}

In recent years, we see a growing availability of devices with computer capabilities. Along with this advent, comes out an opportunity to develop and deploy applications that explore those devices in dynamic environments. However, such environments are inherently characterized by uncertainty, in particular from the perspective of the system designer. To design dependable solutions for environments with a high level of uncertainty, we need models at runtime to represent the system structure, requirements as well as the system's contexts of operation in an integrated way. In addition, we need methods to reason about the system levels of operation and change them at runtime, whenever needed. In this work, we propose to address those issues by devising a component-based model approach for self-adaptation relying on GORE (goal-oriented requirements engineering) and a component-based architecture model. By these means, we plan to develop a middleware that follows and implements that model. Last, but not least, the middleware we will provide fault tolerance strategies to build a foundation for dependable systems.


  %We propose a model based on Goal-Models and component-based architecture in the direction of fulfill this need. We will also develop a middleware that will keep that model. On top of that middleware we will provide fault tolerance strategies to build a foundation for dependable systems.
%%%%%%%%%
  % In a non-adaptive computer systems the computer is static linked to its function in the system.  To achieve a hight level of dependability in such systems one would either rely on right dependable units or on a hight level of redundancy. Both this solutions genearly leads to expensive systems. Appling self-adaptiveness is possible to a system tolerate to failures without a proibit level of redundancy. allow the development of dependable multi-processor heterogeneous systems in face of not dependable units with a lower level of redundancy.

  % Our approach is a middleware that permit uses a  multi-agent runtime goal model, in witch agents can accomplish goals by selecting strategies in its local strategies repository, evaluating them by a utility function. On top of this we aim at easy the development of adaptable open-systems applications, allowing runtime discovery of peers and opportunistically sharing tasks between them, easy integration of new functional strategies and easy integration of new adaptation strategies.


\end{abstract}

%\selectlanguage{brazil}
\tableofcontents
\listoffigures
%listoftables

\textual

\chapter{Introduction}
%!TEX root = /Users/gabrielsr/Dropbox/Projeto Final/monografia/monografia.tex


\section{Problem Definition}

Lalala \cite{stodden_best_2014}
%\section{Problem Definition}

% is there any component model suitable to trace goals and component service.

To allow the system make decisions about its structure based on requirements and context we need a model that can correlate this 3 concepts: the system, the requirements and context.

What leads us to or general question:

\setlength{\fboxsep}{10pt}
\noindent\fbox{%
    \parbox{0.95\textwidth}{%
        \textbf{Research Question 1 (RQ1):} What would be a good model of software system that
        could allow for reason about the system structure, context and trace the requirements at runtime? In another words, represent the system requirements, structure, operation context and the relationship between this elements?
    }%
}\bigskip

As this question is too difficult to answer directly we will make a proposal of solution and evaluate it. In this work we choose to represent requirements in a Goal Model based approach, the system structure from an architectural point of view and the context as a data reference resolution process.

\setlength{\fboxsep}{10pt}
\noindent\fbox{%
    \parbox{0.95\textwidth}{%
        \textbf{Research Question 2 (RQ2):} Would a model that represent system requirements at runtime as goals, system organization at a component level and a context as a data reference resolution process a model that satisfy RQ1?
    }%
}\bigskip

Yet in relation to RQ1, to develop what would be a good model we did more questions in relation to the fitness of the model for the purpose of decide on system adaptations. The following is in direction of find i a given configuration of the system is valid.

\setlength{\fboxsep}{10pt}
\noindent\fbox{%
    \parbox{0.95\textwidth}{%
        \textbf{Research Question 3 (RQ3):} How to, using the model from RQ1, verify if systems goals are achievable at runtime in face of deployment (configuration) uncertainty.

        Would it be possible to verify ...?

        What model would allow us to verify if systems goals are achievable at runtime in face of deployment (configuration) uncertainty.
    }%
}\bigskip

Beside check the system validity in other to forecast faults we want to be able to tolerate faults. What leads to the next question:

\setlength{\fboxsep}{10pt}
\noindent\fbox{%
    \parbox{0.95\textwidth}{%
        \textbf{Research Question 4 (RQ4):} How to assure that goals instances are always achieved, if they are achievable, in face of components faults ?
    }%
}\bigskip



This research question is the search to insert fault-tolerance, to achieve dependability of the system in face of not dependable components. To achieve a greater level of manutenability, the faul-tolerance techniques should be portable.


\setlength{\fboxsep}{10pt}
\noindent\fbox{%
    \parbox{0.95\textwidth}{%
        \textbf{Research Question 5 (RQ5):}	Is it feasible to implement fault tolerance techniques (\textit{retry, retry on alternate resource, check-
point/restart and replication}) as portable components that plug in the runtime model of the system?
    }%
}\bigskip



We also want to asses if the model is a practical solution.

\setlength{\fboxsep}{10pt}
\noindent\fbox{%
    \parbox{0.95\textwidth}{%
        \textbf{Research Question 6 (RQ6):} The model could be used in a real world application, allowing the system to asses itself capacity of fulfill its requirements, self-heal in case of not and tolerate to faults?
    }%
}\bigskip



\setlength{\fboxsep}{10pt}
\noindent\fbox{%
    \parbox{0.95\textwidth}{%
        \textbf{Research Question 7 (RQ7):} How to engage the community in a common experiment setup for self-adaptation, especially for dependability attributes, so we can compare different approaches ?
    }%
}\bigskip


\setlength{\fboxsep}{10pt}
\noindent\fbox{%
    \parbox{0.95\textwidth}{%
        \textbf{Research Question 8 (RQ8):} Is it feasible to create an integrated model process for adaptive software by bridging goal-oriented requirements engineering with architecture-based adaptation?
    }%
}\bigskip



% até quanto? qualitative / quantitative
% first: traceability model
How can we trace runtime component-systems from goal oriented perspective
what would be a suitable model to represent the traceability between GORE and CBSA?
% Model


% component model that satisfy



% analise and act?


% Define architecture,

%

%
\section{Contributions Summary}

\begin{enumerate}
\item conceptual model for a component-based runtime goal model
  %\begin{itemize}
  %  \
  %\end{itemize}
\item component-based runtime goal model middleware architecture and development

% \item component-based runtime goal model middleware support for evolvable systems

% \item a model for open-adaptation and opportunistic computing using multi-agent and component-based runtime goal model

\end{enumerate}

%\section{Research Questions}

% is there any component model suitable to trace goals and component service.

In order to allow the system make decisions about its structure based on requirements and context we need a model that can correlate these three concepts: the system structure, the goals and the execution context.

% What leads us to or research question:


\setlength{\fboxsep}{10pt}
\noindent\fbox{%
    \parbox{0.95\textwidth}{%
        \textbf{Research Question 1 (RQ1):} What would be a good model of software system that
        could allow one to reason about the system structure, context and trace the goals at runtime? In other words, how to represent the system goals, architectural structure, operation context and their relationship?
    }%
}\bigskip
Previous work in the literature \cite{pessoa_dependable_2015} have partially tackled this research question. However, our major concern here relates to the traceability between goals and tasks, software architectural components and the context of operation.

To develop what would be a good model we address further questions in relation to the fitness of the model for the purpose of deciding on system adaptations. First of all, we want to realize a valid deployment of the system:

\setlength{\fboxsep}{10pt}
\noindent\fbox{%
    \parbox{0.95\textwidth}{%
        \textbf{Research Question 2 (RQ2):} How to, using the model from RQ1, deploy a Self-Adaptive System that is dependable in face of context variability? % verify if systems goals are achievable at runtime in face of deployment (configuration) uncertainty.

    }%
}\bigskip

Beside checking the system validity in other to predict system dependability, we want to be able to tolerate faults at runtime. This leads us to the next research question:

\setlength{\fboxsep}{10pt}
\noindent\fbox{%
    \parbox{0.95\textwidth}{%
        \textbf{Research Question 3 (RQ3):} How to guarantee that the systems is dependable in face of error prone components at runtime?
        % assure that goals are always achieved, if they are achievable, in face of components faults ?
    }%
}\bigskip

% This research question is the search to insert fault-tolerance.

\section{General Objectives}
\begin{itemize}
  \item propose, implement and validate a goal-driven platform in witch
\end{itemize}

\section{Specific Objectives}
\begin{itemize}
  \item propose a conceptual model for component-based runtime goal-model system
  \item implement and validate a middleware for component-based runtime goal model
  %\item implement and validate runtime strategies for evolving the system at runtime
  %\item implement and validate runtime strategies for multi-agent collaboration as open-system support
  \item release the middleware as a comprehensible open source project
  \item implement and validate runtime strategies for dependability in the proposed platform
\end{itemize}


\chapter{Background}

%\section{Self-Adaptive Systems}

Self-adaptive systems have been accepted as a promising approach to tackle context change. Self-adaptivesses is an approach in which the system
\textit{"evaluates its own behavior and changes behavior when the evaluation indicates that it is not accomplishing what the software is intended to do, or when better functionality or performance is possible."}\cite{laddaga_self_1997}.

Self-adaptive software aims to adjust various artifacts or attributes in response to changes in the self and in the context of a software system\cite{salehie_self-adaptive_2009}.

A key concept in self-adaptive systems is the awareness of the system. It has two aspects\cite{salehie_self-adaptive_2009}:
\begin{itemize}
   \item \textit{self-awareness} means a system is aware of its own states and behaviors.
   \item \textit{context-awareness} means that the system is aware of its context,
\end{itemize}

Schilit et al.\cite{klein_survey_2008} define \textit{context} as the sufficiently exact characterization of the situations of a system by means of perceivable information that is relevant for the adaptation of the system

Schilit et al.\cite{klein_survey_2008} define \textit{context adaptation} as \say{a system’s capability of gathering information about the domain it shares an interface with, evaluating this information and changing its observable behavior according to the current situation}.



% laddaga1997: it should relies on software informed about its mission and about its construction and behavior.  This implies that the software has multiple ways of accomplishing its purpose, and has enough knowledge of its construction to make effective changes at runtime.

% Such software should include functionality for evaluating its behavior and performance, and the ability to replan and reconfigure its operations in order to improve its operation.  Self adaptive software should also include a set of components for each major function, along with descriptions of the components, so that components of systems can be selected and scheduled at runtime, in response to the evaluators.

% It also requires the ability to impedance match input/output of sequenced components, and the ability to generate some of this code from specifications. In addition, we seek this new basis of adaptation to be applied at runtime, as opposed to development/design time, or as a maintenance activity.


% mape-k

% different approaches ref salehie

% challenges

%\section{Goal Model}



\section{Runtime Goal Model}
Early requirements models express what are the stakeholder goals and needed functionality. This models are a tool for common communication with various stakeholders by abstracting details that are irrelevant at this level\cite{dalpiaz_runtime_2013}.

Dalpiaz et al.\cite{dalpiaz_runtime_2013} describe a method for extend Design-time Goal Models (DGMs) to create Runtime Goal Models (RGM). RGMs can be used to analyze the system's runtime behavior.

At runtime, system behavior is characterized by \textit{events}, related to \textit{goal instances}.

%\input{background/arch_based_adapt}
\section{Dependability}
Dependability can be defined as the ability of a system to avoid faults in its services
that (1) are more frequent or (2) more severe that is acceptable. Or as the characteristic of a system to be justifiably trusted.

A common terminology used for system deviations as the following: \cite{avizienis_basic_2004}

\begin{itemize}
  \textbf{failure}: or \textif{service failure} is a perceived deviation from the correct service provided by a system.
  \textbf{error}: is a deviation of correct internal system state that can lead to its subsequent failure.
  \textbf{fault}: is the adjudged or
hypothesized cause of an error
\end{itemize}

Dependability include the following attributes:\cite{avizienis_basic_2004}
\begin{itemize}
  \item \textbf{availability}: readiness for correct service.
  \item \textbf{reliability}: continuity of correct service.
  \item \textbf{safety}: absence of catastrophic consequences on the
user(s) and the environment.
  \item \textbf{integrity}: absence of improper system alterations.
  \item \textbf{maintainability}: ability to undergo modifications and repairs.
\end{itemize}


Many means have been developed of how attain the attributes of dependability. This means can be classified as:

\begin{itemize}
  \item \textbf{Fault prevention} means to prevent the occurrence or introduction of faults.
  \item \textbf{Fault tolerance} means to avoid service failures in the presence of faults.
  \item \textbf{Fault removal} means to reduce the number and severity of faults.
  \item \textbf{Fault forecasting} means to estimate the present number, the future incidence, and the likely consequences of faults.
\end{itemize}


% TODO dependability in face of uncertainty

\section{Attain Dependability at Runtime }
To keep dependability in face of uncertainty in the deployment environment some techniques was proposed for runtime analysis at runtime.

Felipe et al\cite{guimaraes_framework_2013} propose a method of fault-tolerance for a scientific workflow execution in grid.

Alessandro Leite \cite{ferreira_leite_user_2014} propose a fault tolerance schema for cloud deployment based on which a fault instance in the cloud is monitored and in case of failure the instance can be restarted or terminated and them a new instance created.

Danilo et al\cite{mendonca_dependability_2015} propose a methodology for fault forecasting by which developer, at design time, annotate the goal decomposition in goal model and specify context variables. A special tool generate a formula for, given a context, evaluate the probability of achieve a goal at runtime.

% Pessoa et al \cite{pessoa_dependable_2015} propose a ... with and evaluate it with focus on safety ...


 % reduce redundancy

\section{Self-Adaptive Systems}

Self-adaptive systems have been accepted as a promising approach to tackle context change. Self-adaptivesses is an approach in which the system
\textit{"evaluates its own behavior and changes behavior when the evaluation indicates that it is not accomplishing what the software is intended to do, or when better functionality or performance is possible."}\cite{laddaga_self_1997}.

Self-adaptive software aims to adjust various artifacts or attributes in response to changes in the self and in the context of a software system\cite{salehie_self-adaptive_2009}.

A key concept in self-adaptive systems is the awareness of the system. It has two aspects\cite{salehie_self-adaptive_2009}:
\begin{itemize}
   \item \textit{self-awareness} means a system is aware of its own states and behaviors.
   \item \textit{context-awareness} means that the system is aware of its context,
\end{itemize}

Schilit et al.\cite{klein_survey_2008} define \textit{context} as the sufficiently exact characterization of the situations of a system by means of perceivable information that is relevant for the adaptation of the system

Schilit et al.\cite{klein_survey_2008} define \textit{context adaptation} as \say{a system’s capability of gathering information about the domain it shares an interface with, evaluating this information and changing its observable behavior according to the current situation}.



% laddaga1997: it should relies on software informed about its mission and about its construction and behavior.  This implies that the software has multiple ways of accomplishing its purpose, and has enough knowledge of its construction to make effective changes at runtime.

% Such software should include functionality for evaluating its behavior and performance, and the ability to replan and reconfigure its operations in order to improve its operation.  Self adaptive software should also include a set of components for each major function, along with descriptions of the components, so that components of systems can be selected and scheduled at runtime, in response to the evaluators.

% It also requires the ability to impedance match input/output of sequenced components, and the ability to generate some of this code from specifications. In addition, we seek this new basis of adaptation to be applied at runtime, as opposed to development/design time, or as a maintenance activity.


% mape-k

% different approaches ref salehie

% challenges

\section{Software Components and Architecture}

Heineman define \textit{software component} as a
\say{software element that conforms to a component model and can be independently deployed and composed without modification according to a composition standard}\cite{heineman_component-based_2001}.

Software components is a unit of composition. Software systems are build by composing different components.  Software components must conform to a component model by having contractually specified interfaces and explicit context dependencies only.\cite{szyperski_component_2002}.

A \textit{component	interface} \say{defines a set of component functional properties, that is, a set of actions that’s understood by both the interface provider (the component) and user (other components, or other software that interacts with the provider)}\cite{crnkovic_software_2011}.
A component interface has a role as a component specification and also a means for interaction between the component and its environment.
A \textit{component model} is a set of standards for a component implementation. These standards can standardize naming, interoperability, customization, composition, evolution and deployment.\cite{heineman_component-based_2001}

The \textit{component deployment} is the process that enables component integration into the system. A deployed component is registered in the system and ready to provide services \cite{crnkovic_software_2011}.

\textit{Component Binding} is the process that connects different components through their interfaces and interaction channels.

Software architecture deals with the definition of components, their external behavior, and how they interact.\cite{kaur_component_2010}

Component based software engineering (CBSE) approach consists in building systems from components as reusable units and keeping component development separate from system development\cite{crnkovic_software_2011}.

CBSE is built on the following four principles\cite{crnkovic_software_2011}:
\begin{itemize}
  \item Reusability. Components, developed once, have the potential for reuse many times in different applications.
  \item Substitutability. Systems maintain correctness even when one component replaces another.
  \item Extensibility. Extensibility aims to support evolution by adding new components or evolving existing ones to extend the system’s functionality.
  \item Composability. System should supports the composition of functional properties (component binding). Composition of extra functional properties, for example composition of components’ reliability, is another possible form of composition.
\end{itemize}

\section{Goal-oriented requirements engineering}

Goal-oriented requirements engineering (GORE) is concerned with the use of goals for eliciting, elaborating, structuring, specifying, analyzing, negotiating, documenting, and modifying requirements\cite{van_lamsweerde_goal-oriented_2001}.

GORE models are the main tool used by system analysts and stakeholders to reason about the system requirements. Goal modeling represents a shift in relation to traditional software development approaches as it focus on stakeholder goals and states that the system needs to achieve and not in how it achieves it\cite{ali_goal-based_2010}. Goal models are graphs representing AND/OR-decomposition of abstract goals down to operationalisable leaf-level goals. \cite{morandini_operational_2009}

A goal is an objective the system under consideration should achieve. \cite{van_lamsweerde_goal-oriented_2001}

\section{TROPOS}
Tropos\cite{bresciani_tropos:_2004} is a methodology for develop multi-agent systems that uses goal models for requirement analyses. Tropos encompasses the software development phases, from Early Requirements to Implementation and Testing.

\subsection{The Tropos key concepts}

The methodology adopts the i* \cite{yu_modelling_1996} modeling framework, which proposes the concepts of actor, goal, task, resource and social dependency to model both the system-to-be and its organizational operating environment\cite{bresciani_tropos:_2004}. In more recent publication \cite{morandini_tropos_2014} about the Tropos modeling framework the concept of \textit{task} was renamed to \textit{plan}.

The following are the key concepts in the Tropos metamodel\cite{morandini_tropos_2014}:

\begin{itemize}
    \item Actor: an entity that has strategic goals and intentionality
    \item Goals: it represents actors’ strategic interests. \texit{Hard goals} are goals that have clear-cut criteria for deciding whether they are satisfied or not. \textit{Softgoals} have no clear-cut criteria and are normally used to describe preferences and quality-of-service demands.

    \item Plan: it represents, at an abstract level, a way of doing something. The execution of a plan can be a means for satisfying a goal or for \textit{satisficing} (i.e. sufficiently satisfying) a softgoal.

    \item Resource: it represents a physical or an informational entity.

    \item Dependency: its a relationship between to actors that specify that one actor (the \textif{depended}) have a dependency to other actor (the \textit{dependee}) to attain some goal, execute some plan or deliver a resource. The object of the dependence is the \texit{dependum}.

    \item Capability: it represents both the \textit{ability} of an actor to perform some action and the \textit{opportunity} of doing this.

\end{itemize}

%\input{background/programming/inversion_of_control}


\chapter{Proposal}
\section{Proposed Solution}

We propose a component model for goal oriented multi-agent systems and a runtime model as a solution for couple with computing environment uncertainty.

This model will be implemented by a middleware, that will keep the traceability between goals and architecture and allow for managing the architecture.

In addition we pretend to implementing fault-tolerance techniques as reusable components.

\section{Conceptual Model}
\label{conceptual_model}

\subsection{Component Model}

We propose and extension to Tropos Model with a component model. By this we aim at creating an appropriate abstraction to allow composable architecture while keeping the traceability between the requirements and implementation at runtime.

Or component model is build around the concept of \textit{strategy}. The strategy at goal model level is a mean of achieving a particular goal. It can have the following realization at runtime:

\begin{itemize}
  \item a strategy can be implemented by a component in the architecture.
  \item can be a delegation to another known agent as a runtime decision.
  \item can be a fault tolerant proxy that combines a implementation or delegation with a fault tolerance technique.
\end{itemize}

From a goal model, each goal will originate at least a capability and strategy. Each OR decomposition of a goal, in the goal model, should correspond to an additional strategy.
An agent in the system has a repository of strategies.

The main concepts are:

\begin{itemize}
  \item \textbf{Agent}: agent is an independent computational unit that manages its own resources (CPU, memory, disk, sensors, etc). After system deployment an agent is turned into an actor of a Tropos model.
  \item \textbf{Capability}: description of a kind of goal that an agent can perform. Its a component interface description in the architecture.
  \item \textbf{Strategy}: a strategy is an alternative way to achieve a goal.
  \item \textbf{Strategy Repository}: an agent has a repository of its known strategies.
\end{itemize}

\begin{figure}
  \centering
  \includegraphics[width=\linewidth]{goalp-agent-repo-rcm-depl}
  \caption{Agent Respresentation}
  \label{fig:goalp-agent}
\end{figure}

A strategy should declare its functional interface by declaring which capability it implements. A strategy also needs to declare its dependencies as plans and capabilities that it depends on. A strategy is active if all its dependencies are currently satisfied. It is inactive otherwise. A capability is satisfied if an agent has at least one active strategy that implement that capability.

\subsection{Architectural Layers}

For the implementation of the conceptual model, we propose an architecture of three layers:

\begin{itemize}
  \item \textbf{Infrastructure}: This layer is responsible for the component model implementation, how capability, strategies, plans and goals instances are described. It also responsible for agent and goals life cycle, strategies management and selection.
  \item \textbf{Adaptation}: This layer is responsible for keep the level of service. It contains strategies to discover peers, team up with peers, gather information to improve strategy selection, etc.
  \item \textbf{Application}: This layer if responsible for functional strategies. The functional strategies are the one that implement user application.
\end{itemize}

\subsection{Runtime Model}

Dalpiaz et al. \cite{dalpiaz_runtime_2013} argue that traditional goal models are not enough for reason about the system at runtime. They propose a distinction between Desgin-time Goal Model (DGM) and Runtime Goal Model (RGM). We extend their proposal of RGM with a component model and runtime strategy selection.

\begin{itemize}

\item \textbf{Goal Instance}: an actual instance of an objective for a given data set.

\item \textbf{Believe}: is the model of an actor about itself and the context. Represents what the agent know.

\item \textbf{Strategy Deployment}: occurs after the strategy selection, is the the binding between the strategy, its capabilities dependencies and its environment dependencies.

\end{itemize}

In the proposed model an actor achieve a goal by \emph{deploying a strategy} and executing it. For instance the deployment of a strategy is a capability itself.

\subsection{Strategy Selection and Deployment}

In order to allow component based adaptation we propose a mechanist of strategy selection at runtime. This mechanism is part of the infrastructure layer. For a hight level of flexibility we propose that the selection mechanism itself should be implemented by components in the architecture.

At a low level an agent has the capability of fulfill goals. Inspired by component based frameworks like Rainbow\cite{garlan_rainbow:_2004} we propose that the strategy should be chosen by means of a utility function. That utility function should be responsible to calculate which available strategy will have a better contribution for softgoals.

After a strategy is selected it is deployed. Strategy deployment corresponds to a component binding in CBSE.

The capability <fulfill goal> and its corresponding strategy should be as follows:

\begin{figure}
  \centering
  \includegraphics[width=\linewidth]{strategy_deployment}
  \caption{The deployment of a strategy}
  \label{fig:agent_composition}
\end{figure}

\begin{itemize}
\item \textbf{<Fulfill Goals>} the capability to fulfill generic goals.

\item \textbf{<Fulfill Goals>} strategy to fulfill goals by selecting available strategies and evaluating them with an utility function. Consists of 3 sub-goals:
  \begin{itemize}
    \item Find Matching Strategies
    \item Decide on Strategies
    \item Deploy the Selected Strategy
  \end{itemize}
\end{itemize}

And the following three strategies implement the previous 3 capabilities.

\begin{itemize}
  \item \textbf{<Find Local Matching Strategies>} accomplishes <Find Matching Strategies>
  returns the list of matching strategies. A matching strategy is any strategy that implements the goal interface.

  \item \textbf{<Select a Strategy>} accomplishes <Decide on Strategies>
  using a pre-configured utility function that analyses strategies metadata and select a strategy.

  \item \textbf{<Deploy Strategy>} accomplishes <Deploy the Selected Strategy>. Bind the components with the goal instance and dependencies.
\end{itemize}

\subsection{Awareness}

The agent have a model of that repository that it can use for reason about its capabilities, strategies and plans. The agent is also able to manage its own repository.

Self-awareness is provided by a set of self-awareness strategies. Example of self-aware strategies are:

\begin{itemize}
  \item \textbf{<List Local Strategies>}, how the agent can know it actual capabilities.

  \item \textbf{<List Goal Instances>}, how the agent can know it current intentions.

  \item \textbf{<List Deployed Strategies>}, how the agent can know it current running strategies.

\end{itemize}


\subsection{Multi Agent Deployment}

In order to support multi-agent deployments we will extend the conceptual model and create a set of strategies at the adaptation level. First, in the conceptual model, we propose a extension of the Tropos model concept of \textit{role}.

A role, in runtime, will be associated with a set of capabilities. A role is interpreted also as responsibilities of an actor. So if an agent accepts a role it should keep active strategies to accomplish goals related to capabilities associated with its role.
We will call the state when an actor is satisfying all its assigned roles as \textit{satisfying actor}.

The deployment is the process of assign all needed roles to available agents and moving them to a state of satisfying actors.

In order to allow a system to adapt to deployment variability we propose an \textit{actor verification strategy}. With this strategy the actor should check if itself is currently a satisfying actor. If there is a capability that the agent is currently not satisfying it can use \textit{recovery strategies}.

Recovery strategies could be:
\begin{itemize}
  \item lookup for strategy in external strategy repository;
  \item redistribute roles with peers;
  \item lookup for a peer with the missing capability;
  \item restore resource availability;
  %\item degradate the system, making the capability known as not available
\end{itemize}

\subsection{Virtual Strategies}

To extend the model with adaptation capabilities we propose \textif{Virtual Strategies} (VS). VS are means of achieving a goal that is not by binding a component. We propose two kinds of VS:

\begin{itemize}
  \item \textbf{delegating strategies}. VS that allow an agent to achieve a goal by delegating it to a peer agent. When a agent discover that a peer have some capabilities it creates proxy strategies in its own repository that allow it to fulfill goals by delegating to the peer.
  \item \textbf{fault tolerance strategies}. VS that wrap other strategies in a fault tolerance scheme. Examples of possible fault tolerance strategies are \textit{active replication} and \textit{retry}.
\end{itemize}

The creation of virtual strategies will be performed by \textit{Virtual Strategy Producers} (VSP) that are strategies in itself. The creation of a VSP is a \textit{Virtual Strategy Instance} VSI.

A VSI has its own metadata and is evaluated by the utility function in the process of strategy selection.
%TODO For example, if agent has a goal G1, that needs capability C1 to be accomplish and this capability is implemented by strategies S1, S2. An it has also an \textit{active replication VSP} it will chose between deploying S1, S2 or (S1 || S2). In the example, (S1 || S2) been a VSI, the result of active replication of S1 and S2. With this we expect to allow for the development of portable fault tolerance strategies.

% TODO  Strategies and Virtual Strategies can be combined to make flexible solutions at runtime. For example if an actor has a Goal G1, that needs capability C1 to be accomplish and this capability is implemented by strategies S1, S2, S3.

 A VSP can reason about the context in order to adapt the strategy creation. So, for example, in a context that the agent needs to save battery, the active replication strategy may not be produced.

\section{Related Work}
\label{related}


Rainbow is a framework for self-adaptation architecture based\cite{garlan_rainbow:_2004}. It keeps an model of the architecture of the system and can be extended with rules to analyses the system behavior at runtime, find adaptation strategies and perform this changes. It separate the functional  code (internal mechanisms) from adaptation code (external mechanism) in a schema called external control, influenced by control theory. \cite{garlan_software_2009}
Different from our proposal Rainbow don't enforce an specific architecture what could be special useful in case of retrofitting a pre-existent systems. Different from our proposal its not goal-oriented an for the best of our knowledge there is no work on how to map goals to  components.

MUSIC project provides a component-based middleware for adaptation that propose to separate the self-adaptation from business logic and delegate adaptation logic to generic middleware. As in our propose it adapts buy evaluate in runtime the utility of alternatives, to chose a feasible one (e.g., the one evaluated as with highest utility)\cite{rouvoy_music:_2009}. As Rainbow, MUSIC is not goal-oriented.
% Yet it provide means of supporting seamless configuration of component frameworks based on local, remote components and services.

Salehie et al. \cite{salehie_towards_2012} propose a run-time goal model and its related
action selection. It models adaptable software as a system that exposes sensors and effectors and  proposes a model consisting in Goals, Attributes and Action for selecting actions that will effect the adaptable software at runtime, giving sensed attributes.
So the adaptation mechanism is to choose the best action given the actual attributes.
As this work it uses explicit runtime goals and make them visible and traceable.
Different from it we use a more symmetric approach that can allow for functional
and adaptation management.
%The validation is make on simulated environment.

Günalp et al. \cite{gunalp_autonomic_2012} propose a middleware for pervarsive software with autonomic capabilities. The approach is service based. It proposes a component written in a custom language and the use of components repository that allows the discovery on new sensors. The system present a support for adaptability by using policies.
% Different from this work it did not present support for collaboration/delegation of goals to another peers. It don't allow for runtime incorporation of adaptation strategies, also.


\chapter{Methodology}
In this section we will describe our methodology. At a hight level our main activities will be: review the literature, elaborate a conceptual model, implement and evaluate a middleware, write the dissertation, write paper and viva presentation.
The implementation will be divided in 2 phases so we can submit intermediate results to scientific conferences.

\begin{enumerate}

\item \emph{literature review}

First we will conduct a literature review with the objective to find all the relevant related works, what is the state of the art, review that works and better proposition our proposal in relation to that works.

\item \emph{elaborate a conceptual model and architecture}

In parallel to the literature review, we will develop a conceptual model to integrate Goal-Oriented Requirements Engineering, Architecture-Based Self-Adaptation so that the system goals can be traced to architecture modules at runtime. This conceptual-model will describe the Architectural components of the proposed architecture, their relations, how this elements are related to the goal model that they implement and how this model can be manipulated at runtime.


\item \emph{implement and validate a middleware (Core)}

At this stage we will implement the core of a middleware to support execution of components as described by the proposed conceptual model and evaluate. To do the evaluation we will specify and describe a study case system and implement the system using the middleware and the proposed architecture.

\item \emph{write paper about the middleware (Core)}

We will report the results of the middleware implementation and validation as a paper.

%\item \emph{implement and validate evolution capacities in the middleware}

%implement and evaluate strategies for runtime strategies incorporation. Report the results.

%\item \emph{implement and validate multi-agent strategies}

 %implement and evaluate strategies for support communication and discovery of agents (multi-agent) as a support for open-systems. Report the results.

\item \emph{implement and validate dependability strategies}

 implement and evaluate GODA\cite{mendonca_dependability_2015} strategy for runtime dependability analysis at runtime.

 \item \emph{write paper about dependability strategies}

 We will report the results of the middleware implementation and validation as a paper.

\item \emph{write up dissertation}

gather all intermediate results and format as a dissertation.

\item \emph{viva}

present the dissertation.

\end{enumerate}

\chapter{Expected Results}
\section{Expected Results}

With this work we expect to:

\begin{itemize}
\item Collaborate to self-adaptation corpus of knowledge by contributing with a conceptual model that map concepts of Goal-Model and Architecture-based adaptation.

\item Generate a propose of architecture and reference middleware for development of applications that face a high level of uncertainty and couple with this uncertainty by reasoning about its goal model at runtime.

\item By the former two items, we expect to allow for future development of more flexible and dependable software systems.

\end{itemize}

\chapter{Chronogram}
The proposed work is composed of the following activities:

\begin{enumerate}
\item \emph{define scope}
\item \emph{conduct specific research}
\item \emph{elaborate a conceptual model}
\item \emph{implement and validate a middleware (Core)}
% \item \emph{implement and validate runtime strategies incorporation mechanism in the middleware}
%\item \emph{implement and validate multi-agent strategies}
% implement and evaluate strategies for support multi-agent and open-systems. Report the results.

\item \emph{implement and validate dependability strategies}

implement and evaluate GODA\cite{mendonca_dependability_2015} strategy for runtime dependability analysis at runtime. Report the results.

\item \emph{write dissertation}

gather all intermediate results, review the document.

\item \emph{defence}

defence of the dissertation.

\end{enumerate}

\begin{table}[htbp]
\tiny
\begin{flushleft}
\begin{tabular}{|p{1.1cm}|p{0.75cm}|p{0.75cm}|p{0.75cm}|p{0.75cm}|p{0.75cm}|p{0.75cm}|p{0.75cm}|p{0.75cm}|p{0.75cm}|p{0.75cm}|p{0.75cm}|}
\hline
\textbf{Activity}&	\textbf{prior Aug/15}&	\textbf{Sep/15}&	\textbf{Out/15}&	\textbf{Nov/15}&	\textbf{Dez/15}&	\textbf{Jan/16}&	\textbf{Fev/16}&	\textbf{Mar/16}&	\textbf{Apr/16}&	\textbf{May/16}&	\textbf{Jun/16}\\
\hline	1			&	X	&			&	 		&			&		&		&	 		&	 		&	 	&		 &				\\
\hline	2			&		&	 X	&			&			&		&		&	 		&	 		&	 	&		 &				\\
\hline	3			&		&	 X	&			&			&		&		&	 		&	 		&	 	&		 &				\\
\hline	4			&		&	  	&	 X	&	X	  &	X	&		&	 	  &	   	&	 	&		 &				\\
\hline	5			&		&	 		&	 		&			&		&	X	&	 X	&	 X	&	  &		 &				\\
%\hline	6			&		&	 		&	 		&			&		&		&	 X	&	   	&		&		 &				\\
%\hline	7			&		&	 		&	 		&			&		&		&	 		&	 X	&	  &		 &	 			\\
\hline	6			&		&	 		&	 		&			&		&		&	 		&	 		&	X	&	X	 &	X			\\
\hline	7			&		&	 		&	 		&			&		&		&	 		&	 		&	  &		 &	X			\\
\hline
\end{tabular}
\end{flushleft}
\caption{Chronogram of Proposed Activities}
\label{tbcrono}
\end{table}


\postextual
\anexos


\bibliographystyle{plain}
\bibliography{bibliografia}
\end{document}
