Dependability is a omnipresent requirements for software systems, virtually all systems should be dependable.

With increasing popularity of mobile computation and wireless networks we have seen the rise of  interest in new domains of computation that uses the capacity of heterogeneous computer units in a given location.
Example of such domains are Ubiquitous Computing \cite{bell_yesterdays_2007}, Internet of Things (IoT)\cite{atzori_internet_2010}, Assisted Living\cite{kleinberger_ambient_2007} and Opportunistic Computing\cite{smaldone_improving_2011}. In such domains, the computing environment can greatly vary from place to place and is unknown at design-time.
Uncertainty at design-time poses challenge in the development of dependable systems. From the point of manutenability it can be challenging to structure the system in order to properly manage changes and keep different tradeofs.

Need for runtime change.

Salehie at al. \cite{salehie_self-adaptive_2009} define self-adaptive software as one that adjust artifacts or attributes in response to changes in the \textit{self} and in the \textit{context} of a software system.
\textit{Self} being \say{the whole body of software}. And \textit{context} being \say{everything in the operating environment that affects the system's properties and its behavior}.

Architecture-based (or component-based) self-adaptive approaches to implement self-adaptive systems adapt the systems by acting on components of the system or by replacing them\cite{garlan_software_2009}.


Goal Oriented Requirements Engineering (GORE) approaches have gained special attention as a technique to specify self-adaptative systems\cite{morandini_goal-oriented_2009}.
Goals capture, the various objectives the system under consideration should achieve\cite{van_lamsweerde_goal-oriented_2001}.
Dalpiaz et al.\cite{dalpiaz_runtime_2013} proposed Runtime Goal Models to reason about runtime fulfillment of goals.

A multi-agent system (MAS) is a distributed computing system with autonomous interacting intelligent agents that coordinate their actions so as to achieve its goals\cite{woolridge_introduction_2001}.

Autonomous software agents provide a promising solution to the needs of decentralized networked systems, able to adapt their behaviour in a complex and dynamically changing environment \cite{morandini_goal-oriented_2009}.

In this expanding domains software developers and software systems has to deal with uncertainty.




From the point of view of manutenability ...
OO can not be sufficient.

Availability, reliability, availability, safety...

From the point of view of system engineering it can be challenging to foresee and properly address different combination of changes in the system operation.

To handle this complexty we need to create levels of abstraction.



Goal oriented requirements engineering is promissing. We see some advances in goals at
Runtime Goal Models.

From the Software Engineering point of view, goals and agents as abstractions. Yet we need to link it with existing SE practices as components, development process, architecture, software configuration management.

Improving dependability by managing goals at runtime.

This work is in the direction of structure system around goals.

System components should

The system should be structured as agents able to reflect about its goals.


Contributions

Literature review in self-adaptive systems, agents, dynamic product lines, dependability
Associating related concepts.

Extension of Tropos Conceptual Model with Software configuration view (components, artifacts, deployment)



Deployment implementation

Fault-tolerance
