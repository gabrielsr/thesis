%- visão geral do contexto.
We see a growing interest in computing applications that should rely on heterogeneous computing environments. In order to handle some kind of variability, such as two possible types of graphical processors in a desktop computer, we can use simple approaches as a script at deployment-time that chooses the right software library to be copied to a folder.
These simple approachers are centralized and created at design-time. They require one specialist or team to control the entire space of variability.
Such approaches are not scalable to highly heterogeneous environments, like Internet of Things (IoT), where each end user can have a different computing environment with broad range of different resources available.
In such environments it is impossible to predict the computing environment at design-time, implying that deciding on the correct configuration for each environment at design time is impossible.
% leading to impossibility in creating centralized design-time centered models of variability to decide on the correct configuration for each environment.
In our work, we propose Goalp: a method that allows a system to autonomously deploy itself by reflecting about its goals and its computing environment.
We evaluate our approach based on an example of a case study. The algorithm for deployment planning returned the expected response for the tested case.
