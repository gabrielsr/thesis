\documentclass[mestrado]{pacotes/unb-cic}
\usepackage[american,brazil]{babel}
\usepackage[T1]{fontenc}
\usepackage{indentfirst}
\usepackage{natbib}
\usepackage{xcolor,graphicx,url}
\usepackage[utf8]{inputenc}
\usepackage{booktabs} % for tables
\usepackage{verbatim} %% multiline comments
\usepackage{dirtytalk} %% Quotes

\graphicspath{ {imagens/} } % path to images

%\bibpunct[; ]{(}{)}{,}{a}{}{;}%muda colchetes para parenteses

% definicoes previas do documento
%\selectlanguage{brazil}
\title{A Run-Time Goal Model for Self-Adaptation in Open-Systems}

\orientador[a]{\prof[a] \dr[a] Genaina Nunes Rodrigues}{CIC/UnB}

\coordenador[a]{\prof[a] \dr[a] Alba Cristina Magalhaes Alves de Melo}{CIC/UnB}

\diamesano{9}{Junho}{2016}


\membrobanca{\prof }{CIC/UnB}

\membrobanca{\prof }{CIC/UnB}

\autor{Gabriel Siqueira}{Rodrigues}

\CDU{004.4}

\palavraschave{dependabilidade}
\keywords{dependability}

%-------------------------------------------------

\begin{document}

\maketitle

\pretextual
%\begin{agradecimentos}
%Agradecemos à nossa orientadora, \prof[a] \dr[a] Genaina Nunes Rodrigues,
%\end{agradecimentos}

%\begin{resumo}
%Nesse trabalho apresentamos...
%\end{resumo}

\selectlanguage{american}

\begin{abstract}%\textbf{}

  In recent years we see a growing availability of devices with computer capabilities. With this come an opportunity for applications that use a number of heterogeneous computer units in an opportunistic way. The development of such application are currently challenging.

  % In a non-adaptive computer systems the computer is static linked to its function in the system.  To achieve a hight level of dependability in such systems one would either rely on right dependable units or on a hight level of redundancy. Both this solutions genearly leads to expensive systems. Appling self-adaptiveness is possible to a system tolerate to failures without a proibit level of redundancy. allow the development of dependable multi-processor heterogeneous systems in face of not dependable units with a lower level of redundancy.

  Our approach is a middleware that permit uses a  multi-agent runtime goal model, in witch agents can accomplish goals by selecting strategies in its local strategies repository, evaluating them by a utility function. On top of this we aim at easy the development of adaptable open-systems applications, allowing runtime discovery of peers and opportunistically sharing tasks between them, easy integration of new functional strategies and easy integration of new adaptation strategies.

\end{abstract}

%\selectlanguage{brazil}
\tableofcontents
\listoffigures
%listoftables

\textual

\chapter{Introduction}
%!TEX root = /Users/gabrielsr/Dropbox/Projeto Final/monografia/monografia.tex


\section{Problem Definition}

Lalala \cite{stodden_best_2014}
\section{Problem Definition}

% is there any component model suitable to trace goals and component service.

To allow the system make decisions about its structure based on requirements and context we need a model that can correlate this 3 concepts: the system, the requirements and context.

What leads us to or general question:

\setlength{\fboxsep}{10pt}
\noindent\fbox{%
    \parbox{0.95\textwidth}{%
        \textbf{Research Question 1 (RQ1):} What would be a good model of software system that
        could allow for reason about the system structure, context and trace the requirements at runtime? In another words, represent the system requirements, structure, operation context and the relationship between this elements?
    }%
}\bigskip

As this question is too difficult to answer directly we will make a proposal of solution and evaluate it. In this work we choose to represent requirements in a Goal Model based approach, the system structure from an architectural point of view and the context as a data reference resolution process.

\setlength{\fboxsep}{10pt}
\noindent\fbox{%
    \parbox{0.95\textwidth}{%
        \textbf{Research Question 2 (RQ2):} Would a model that represent system requirements at runtime as goals, system organization at a component level and a context as a data reference resolution process a model that satisfy RQ1?
    }%
}\bigskip

Yet in relation to RQ1, to develop what would be a good model we did more questions in relation to the fitness of the model for the purpose of decide on system adaptations. The following is in direction of find i a given configuration of the system is valid.

\setlength{\fboxsep}{10pt}
\noindent\fbox{%
    \parbox{0.95\textwidth}{%
        \textbf{Research Question 3 (RQ3):} How to, using the model from RQ1, verify if systems goals are achievable at runtime in face of deployment (configuration) uncertainty.

        Would it be possible to verify ...?

        What model would allow us to verify if systems goals are achievable at runtime in face of deployment (configuration) uncertainty.
    }%
}\bigskip

Beside check the system validity in other to forecast faults we want to be able to tolerate faults. What leads to the next question:

\setlength{\fboxsep}{10pt}
\noindent\fbox{%
    \parbox{0.95\textwidth}{%
        \textbf{Research Question 4 (RQ4):} How to assure that goals instances are always achieved, if they are achievable, in face of components faults ?
    }%
}\bigskip



This research question is the search to insert fault-tolerance, to achieve dependability of the system in face of not dependable components. To achieve a greater level of manutenability, the faul-tolerance techniques should be portable.


\setlength{\fboxsep}{10pt}
\noindent\fbox{%
    \parbox{0.95\textwidth}{%
        \textbf{Research Question 5 (RQ5):}	Is it feasible to implement fault tolerance techniques (\textit{retry, retry on alternate resource, check-
point/restart and replication}) as portable components that plug in the runtime model of the system?
    }%
}\bigskip



We also want to asses if the model is a practical solution.

\setlength{\fboxsep}{10pt}
\noindent\fbox{%
    \parbox{0.95\textwidth}{%
        \textbf{Research Question 6 (RQ6):} The model could be used in a real world application, allowing the system to asses itself capacity of fulfill its requirements, self-heal in case of not and tolerate to faults?
    }%
}\bigskip



\setlength{\fboxsep}{10pt}
\noindent\fbox{%
    \parbox{0.95\textwidth}{%
        \textbf{Research Question 7 (RQ7):} How to engage the community in a common experiment setup for self-adaptation, especially for dependability attributes, so we can compare different approaches ?
    }%
}\bigskip


\setlength{\fboxsep}{10pt}
\noindent\fbox{%
    \parbox{0.95\textwidth}{%
        \textbf{Research Question 8 (RQ8):} Is it feasible to create an integrated model process for adaptive software by bridging goal-oriented requirements engineering with architecture-based adaptation?
    }%
}\bigskip



% até quanto? qualitative / quantitative
% first: traceability model
How can we trace runtime component-systems from goal oriented perspective
what would be a suitable model to represent the traceability between GORE and CBSA?
% Model


% component model that satisfy



% analise and act?


% Define architecture,

%

\section{Proposed Solution}

We propose a component model for goal oriented multi-agent systems and a runtime model as a solution for couple with computing environment uncertainty.

This model will be implemented by a middleware, that will keep the traceability between goals and architecture and allow for managing the architecture.

In addition we pretend to implementing fault-tolerance techniques as reusable components.


\section{Contributions Summary}

\begin{enumerate}
\item conceptual model for a component-based runtime goal model
  %\begin{itemize}
  %  \
  %\end{itemize}
\item component-based runtime goal model middleware architecture and development

% \item component-based runtime goal model middleware support for evolvable systems

% \item a model for open-adaptation and opportunistic computing using multi-agent and component-based runtime goal model

\end{enumerate}

\input{intro/plan_for_completion}

\chapter{Background}

%\section{Self-Adaptive Systems}

Self-adaptive systems have been accepted as a promising approach to tackle context change. Self-adaptivesses is an approach in which the system
\textit{"evaluates its own behavior and changes behavior when the evaluation indicates that it is not accomplishing what the software is intended to do, or when better functionality or performance is possible."}\cite{laddaga_self_1997}.

Self-adaptive software aims to adjust various artifacts or attributes in response to changes in the self and in the context of a software system\cite{salehie_self-adaptive_2009}.

A key concept in self-adaptive systems is the awareness of the system. It has two aspects\cite{salehie_self-adaptive_2009}:
\begin{itemize}
   \item \textit{self-awareness} means a system is aware of its own states and behaviors.
   \item \textit{context-awareness} means that the system is aware of its context,
\end{itemize}

Schilit et al.\cite{klein_survey_2008} define \textit{context} as the sufficiently exact characterization of the situations of a system by means of perceivable information that is relevant for the adaptation of the system

Schilit et al.\cite{klein_survey_2008} define \textit{context adaptation} as \say{a system’s capability of gathering information about the domain it shares an interface with, evaluating this information and changing its observable behavior according to the current situation}.



% laddaga1997: it should relies on software informed about its mission and about its construction and behavior.  This implies that the software has multiple ways of accomplishing its purpose, and has enough knowledge of its construction to make effective changes at runtime.

% Such software should include functionality for evaluating its behavior and performance, and the ability to replan and reconfigure its operations in order to improve its operation.  Self adaptive software should also include a set of components for each major function, along with descriptions of the components, so that components of systems can be selected and scheduled at runtime, in response to the evaluators.

% It also requires the ability to impedance match input/output of sequenced components, and the ability to generate some of this code from specifications. In addition, we seek this new basis of adaptation to be applied at runtime, as opposed to development/design time, or as a maintenance activity.


% mape-k

% different approaches ref salehie

% challenges

%\subsubsection{Architecture}

\begin{figure}[!htb]
  \centering
  \includegraphics[width=\linewidth]{goald_mape_k}
  \caption{Goald Deployment Manager}
\label{fig:goald_mape_k}
\end{figure}

\begin{itemize}
  \item events: are handled to registered listeners.
\end{itemize}

\subsubsection{Knowledge Base}

The knowledge base store a model of goals the system much achieve, the computing environment context and current deployed artifacts.

The information in the knowledge base can be of two types:
\begin{itemize}
  \item facts: can be queried about by registered components
\end{itemize}


\subsubsection{Monitors}

Monitors are components that observes changes in the goals and context. All
\begin{itemize}
  \item Goal Change Monitor is responsible for listening to request of include or remove goals.
  \item Computing Environment monitor is responsible to verify \emph{facts} about the computing enivronment as introduced in~\ref{context}. \emph{Facts} are stored in the Knowledge Base and used to evaluate context conditions.
  \item Repository monitors observe changes in software repositories.
  \item Introspective monitors observe the system state and behavior. (ex: monitor that check if there is any monitor that dispatches events that no one cares about)
\end{itemize}

The result of monitors can be:
\begin{itemize}
  \item Change the knowledge base
  \item Dispatch Events
\end{itemize}

Core Components:
\begin{itemize}
  \item Timer Monitoring Policy:
  \item Computing Environment Sensors: listen to timeouts of monitoring policy. Sense the environment. Update the knowledge base and dispatch events for changes in the environment.
  \item Goals change listeners: listen for external entities that want to change the goals of the machine. The interface is implementation dependent (e.g HTTP service, GUI, command line). Dispatch External Change Requests.
  \item Repository Monitor: queries repository for information about components.
\end{itemize}



\subsubsection{Analyzer}

Objective change analyzer

Computing environment change - evaluate if any system goal is affected.

Evaluate parametric formula for managed goals. If a goal probability of success drops below a threshold, dispatch deployment replanting event.

Handling evolution: simple approach favor a superior version.


\begin{itemize}
  \item Goals Change- Check if a goal request is a change request. A goal removal will affect another goals?

  \item Adaptation
In case of change in the available resources
it should be analyzed if the change threats the achievements of goals.

\end{itemize}

Listens to:
\begin{itemize}
  \item changes in the environment
  \item external changes in goals
\end{itemize}

Dispatch
\begin{itemize}
  \item Change Request: request a change in the deployment. Contains affected goals for what deployment should be replanned.
\end{itemize}

\subsubsection{Planner}

Receive change requests and enqueue it.

Deployment Change Planner is responsible for finding which operation should be executed in order to (1) make the active goals achievable. (2) Free up resource not associated with active goals.

How deployment is planned and the algorithm used will be described in Section~\ref{sec:deployment_planning}.

Components
\begin{itemize}
  \item Context Evaluation: it is responsible to evaluate if context conditions are satisfied for a given component in a given context.
\end{itemize}

Listens To:
\begin{itemize}
  \item Change Request
\end{itemize}

Dispatch:
\begin{itemize}
  \item Query Repositories: request information about available components that provide given goals.
  \item Execute Plan: request a deployment plan to be executed.
\end{itemize}



\subsubsection{Execute}

Executor components are responsible to actuate in the system.
Deployment Change Executor is responsible for get components from repository and execute deployment operations such as install and uninstall components.

% Fix Resource

% Change monitoring policy

Listen To:
\begin{itemize}
  \item Execute Plan:
\end{itemize}

%\input{background/cloud}
%\section{Goal-oriented requirements engineering}

Goal-oriented requirements engineering (GORE) is concerned with the use of goals for eliciting, elaborating, structuring, specifying, analyzing, negotiating, documenting, and modifying requirements\cite{van_lamsweerde_goal-oriented_2001}.

GORE models are the main tool used by system analysts and stakeholders to reason about the system requirements. Goal modeling represents a shift in relation to traditional software development approaches as it focus on stakeholder goals and states that the system needs to achieve and not in how it achieves it\cite{ali_goal-based_2010}. Goal models are graphs representing AND/OR-decomposition of abstract goals down to operationalisable leaf-level goals. \cite{morandini_operational_2009}

A goal is an objective the system under consideration should achieve. \cite{van_lamsweerde_goal-oriented_2001}

\section{TROPOS}
Tropos\cite{bresciani_tropos:_2004} is a methodology for develop multi-agent systems that uses goal models for requirement analyses. Tropos encompasses the software development phases, from Early Requirements to Implementation and Testing.

\subsection{The Tropos key concepts}

The methodology adopts the i* \cite{yu_modelling_1996} modeling framework, which proposes the concepts of actor, goal, task, resource and social dependency to model both the system-to-be and its organizational operating environment\cite{bresciani_tropos:_2004}. In more recent publication \cite{morandini_tropos_2014} about the Tropos modeling framework the concept of \textit{task} was renamed to \textit{plan}.

The following are the key concepts in the Tropos metamodel\cite{morandini_tropos_2014}:

\begin{itemize}
    \item Actor: an entity that has strategic goals and intentionality
    \item Goals: it represents actors’ strategic interests. \texit{Hard goals} are goals that have clear-cut criteria for deciding whether they are satisfied or not. \textit{Softgoals} have no clear-cut criteria and are normally used to describe preferences and quality-of-service demands.

    \item Plan: it represents, at an abstract level, a way of doing something. The execution of a plan can be a means for satisfying a goal or for \textit{satisficing} (i.e. sufficiently satisfying) a softgoal.

    \item Resource: it represents a physical or an informational entity.

    \item Dependency: its a relationship between to actors that specify that one actor (the \textif{depended}) have a dependency to other actor (the \textit{dependee}) to attain some goal, execute some plan or deliver a resource. The object of the dependence is the \texit{dependum}.

    \item Capability: it represents both the \textit{ability} of an actor to perform some action and the \textit{opportunity} of doing this.

\end{itemize}

%\section{Dependability}
Dependability can be defined as the ability of a system to avoid faults in its services
that (1) are more frequent or (2) more severe that is acceptable. Or as the characteristic of a system to be justifiably trusted.

A common terminology used for system deviations as the following: \cite{avizienis_basic_2004}

\begin{itemize}
  \textbf{failure}: or \textif{service failure} is a perceived deviation from the correct service provided by a system.
  \textbf{error}: is a deviation of correct internal system state that can lead to its subsequent failure.
  \textbf{fault}: is the adjudged or
hypothesized cause of an error
\end{itemize}

Dependability include the following attributes:\cite{avizienis_basic_2004}
\begin{itemize}
  \item \textbf{availability}: readiness for correct service.
  \item \textbf{reliability}: continuity of correct service.
  \item \textbf{safety}: absence of catastrophic consequences on the
user(s) and the environment.
  \item \textbf{integrity}: absence of improper system alterations.
  \item \textbf{maintainability}: ability to undergo modifications and repairs.
\end{itemize}


Many means have been developed of how attain the attributes of dependability. This means can be classified as:

\begin{itemize}
  \item \textbf{Fault prevention} means to prevent the occurrence or introduction of faults.
  \item \textbf{Fault tolerance} means to avoid service failures in the presence of faults.
  \item \textbf{Fault removal} means to reduce the number and severity of faults.
  \item \textbf{Fault forecasting} means to estimate the present number, the future incidence, and the likely consequences of faults.
\end{itemize}


% TODO dependability in face of uncertainty

\section{Attain Dependability at Runtime }
To keep dependability in face of uncertainty in the deployment environment some techniques was proposed for runtime analysis at runtime.

Felipe et al\cite{guimaraes_framework_2013} propose a method of fault-tolerance for a scientific workflow execution in grid.

Alessandro Leite \cite{ferreira_leite_user_2014} propose a fault tolerance schema for cloud deployment based on which a fault instance in the cloud is monitored and in case of failure the instance can be restarted or terminated and them a new instance created.

Danilo et al\cite{mendonca_dependability_2015} propose a methodology for fault forecasting by which developer, at design time, annotate the goal decomposition in goal model and specify context variables. A special tool generate a formula for, given a context, evaluate the probability of achieve a goal at runtime.

% Pessoa et al \cite{pessoa_dependable_2015} propose a ... with and evaluate it with focus on safety ...


 % reduce redundancy

%\input{background/model_checking}


\chapter{Related Work}
%\section{Self-Adaptive Systems}

Self-adaptive systems have been accepted as a promising approach to tackle context change. Self-adaptivesses is an approach in which the system
\textit{"evaluates its own behavior and changes behavior when the evaluation indicates that it is not accomplishing what the software is intended to do, or when better functionality or performance is possible."}\cite{laddaga_self_1997}.

Self-adaptive software aims to adjust various artifacts or attributes in response to changes in the self and in the context of a software system\cite{salehie_self-adaptive_2009}.

A key concept in self-adaptive systems is the awareness of the system. It has two aspects\cite{salehie_self-adaptive_2009}:
\begin{itemize}
   \item \textit{self-awareness} means a system is aware of its own states and behaviors.
   \item \textit{context-awareness} means that the system is aware of its context,
\end{itemize}

Schilit et al.\cite{klein_survey_2008} define \textit{context} as the sufficiently exact characterization of the situations of a system by means of perceivable information that is relevant for the adaptation of the system

Schilit et al.\cite{klein_survey_2008} define \textit{context adaptation} as \say{a system’s capability of gathering information about the domain it shares an interface with, evaluating this information and changing its observable behavior according to the current situation}.



% laddaga1997: it should relies on software informed about its mission and about its construction and behavior.  This implies that the software has multiple ways of accomplishing its purpose, and has enough knowledge of its construction to make effective changes at runtime.

% Such software should include functionality for evaluating its behavior and performance, and the ability to replan and reconfigure its operations in order to improve its operation.  Self adaptive software should also include a set of components for each major function, along with descriptions of the components, so that components of systems can be selected and scheduled at runtime, in response to the evaluators.

% It also requires the ability to impedance match input/output of sequenced components, and the ability to generate some of this code from specifications. In addition, we seek this new basis of adaptation to be applied at runtime, as opposed to development/design time, or as a maintenance activity.


% mape-k

% different approaches ref salehie

% challenges

%\input{related/open_system}
%\input{related/model_at_runtime}
%\section{Component Based}

Architectural or component-based modes represent the system as

as a gross composition of components, and their properties of interest\cite{garlan_rainbow:_2004}

considerar the system as

%\section{Dependability}
Dependability can be defined as the ability of a system to avoid faults in its services
that (1) are more frequent or (2) more severe that is acceptable. Or as the characteristic of a system to be justifiably trusted.

A common terminology used for system deviations as the following: \cite{avizienis_basic_2004}

\begin{itemize}
  \textbf{failure}: or \textif{service failure} is a perceived deviation from the correct service provided by a system.
  \textbf{error}: is a deviation of correct internal system state that can lead to its subsequent failure.
  \textbf{fault}: is the adjudged or
hypothesized cause of an error
\end{itemize}

Dependability include the following attributes:\cite{avizienis_basic_2004}
\begin{itemize}
  \item \textbf{availability}: readiness for correct service.
  \item \textbf{reliability}: continuity of correct service.
  \item \textbf{safety}: absence of catastrophic consequences on the
user(s) and the environment.
  \item \textbf{integrity}: absence of improper system alterations.
  \item \textbf{maintainability}: ability to undergo modifications and repairs.
\end{itemize}


Many means have been developed of how attain the attributes of dependability. This means can be classified as:

\begin{itemize}
  \item \textbf{Fault prevention} means to prevent the occurrence or introduction of faults.
  \item \textbf{Fault tolerance} means to avoid service failures in the presence of faults.
  \item \textbf{Fault removal} means to reduce the number and severity of faults.
  \item \textbf{Fault forecasting} means to estimate the present number, the future incidence, and the likely consequences of faults.
\end{itemize}


% TODO dependability in face of uncertainty

\section{Attain Dependability at Runtime }
To keep dependability in face of uncertainty in the deployment environment some techniques was proposed for runtime analysis at runtime.

Felipe et al\cite{guimaraes_framework_2013} propose a method of fault-tolerance for a scientific workflow execution in grid.

Alessandro Leite \cite{ferreira_leite_user_2014} propose a fault tolerance schema for cloud deployment based on which a fault instance in the cloud is monitored and in case of failure the instance can be restarted or terminated and them a new instance created.

Danilo et al\cite{mendonca_dependability_2015} propose a methodology for fault forecasting by which developer, at design time, annotate the goal decomposition in goal model and specify context variables. A special tool generate a formula for, given a context, evaluate the probability of achieve a goal at runtime.

% Pessoa et al \cite{pessoa_dependable_2015} propose a ... with and evaluate it with focus on safety ...


 % reduce redundancy

%\input{related/ecs}
\section{Related Work}
\label{related}


Rainbow is a framework for self-adaptation architecture based\cite{garlan_rainbow:_2004}. It keeps an model of the architecture of the system and can be extended with rules to analyses the system behavior at runtime, find adaptation strategies and perform this changes. It separate the functional  code (internal mechanisms) from adaptation code (external mechanism) in a schema called external control, influenced by control theory. \cite{garlan_software_2009}
Different from our proposal Rainbow don't enforce an specific architecture what could be special useful in case of retrofitting a pre-existent systems. Different from our proposal its not goal-oriented an for the best of our knowledge there is no work on how to map goals to  components.

MUSIC project provides a component-based middleware for adaptation that propose to separate the self-adaptation from business logic and delegate adaptation logic to generic middleware. As in our propose it adapts buy evaluate in runtime the utility of alternatives, to chose a feasible one (e.g., the one evaluated as with highest utility)\cite{rouvoy_music:_2009}. As Rainbow, MUSIC is not goal-oriented.
% Yet it provide means of supporting seamless configuration of component frameworks based on local, remote components and services.

Salehie et al. \cite{salehie_towards_2012} propose a run-time goal model and its related
action selection. It models adaptable software as a system that exposes sensors and effectors and  proposes a model consisting in Goals, Attributes and Action for selecting actions that will effect the adaptable software at runtime, giving sensed attributes.
So the adaptation mechanism is to choose the best action given the actual attributes.
As this work it uses explicit runtime goals and make them visible and traceable.
Different from it we use a more symmetric approach that can allow for functional
and adaptation management.
%The validation is make on simulated environment.

Günalp et al. \cite{gunalp_autonomic_2012} propose a middleware for pervarsive software with autonomic capabilities. The approach is service based. It proposes a component written in a custom language and the use of components repository that allows the discovery on new sensors. The system present a support for adaptability by using policies.
% Different from this work it did not present support for collaboration/delegation of goals to another peers. It don't allow for runtime incorporation of adaptation strategies, also.


\section{Conceptual Model}
\label{conceptual_model}

\subsection{Component Model}

We propose and extension to Tropos Model with a component model. By this we aim at creating an appropriate abstraction to allow composable architecture while keeping the traceability between the requirements and implementation at runtime.

Or component model is build around the concept of \textit{strategy}. The strategy at goal model level is a mean of achieving a particular goal. It can have the following realization at runtime:

\begin{itemize}
  \item a strategy can be implemented by a component in the architecture.
  \item can be a delegation to another known agent as a runtime decision.
  \item can be a fault tolerant proxy that combines a implementation or delegation with a fault tolerance technique.
\end{itemize}

From a goal model, each goal will originate at least a capability and strategy. Each OR decomposition of a goal, in the goal model, should correspond to an additional strategy.
An agent in the system has a repository of strategies.

The main concepts are:

\begin{itemize}
  \item \textbf{Agent}: agent is an independent computational unit that manages its own resources (CPU, memory, disk, sensors, etc). After system deployment an agent is turned into an actor of a Tropos model.
  \item \textbf{Capability}: description of a kind of goal that an agent can perform. Its a component interface description in the architecture.
  \item \textbf{Strategy}: a strategy is an alternative way to achieve a goal.
  \item \textbf{Strategy Repository}: an agent has a repository of its known strategies.
\end{itemize}

\begin{figure}
  \centering
  \includegraphics[width=\linewidth]{goalp-agent-repo-rcm-depl}
  \caption{Agent Respresentation}
  \label{fig:goalp-agent}
\end{figure}

A strategy should declare its functional interface by declaring which capability it implements. A strategy also needs to declare its dependencies as plans and capabilities that it depends on. A strategy is active if all its dependencies are currently satisfied. It is inactive otherwise. A capability is satisfied if an agent has at least one active strategy that implement that capability.

\subsection{Architectural Layers}

For the implementation of the conceptual model, we propose an architecture of three layers:

\begin{itemize}
  \item \textbf{Infrastructure}: This layer is responsible for the component model implementation, how capability, strategies, plans and goals instances are described. It also responsible for agent and goals life cycle, strategies management and selection.
  \item \textbf{Adaptation}: This layer is responsible for keep the level of service. It contains strategies to discover peers, team up with peers, gather information to improve strategy selection, etc.
  \item \textbf{Application}: This layer if responsible for functional strategies. The functional strategies are the one that implement user application.
\end{itemize}

\subsection{Runtime Model}

Dalpiaz et al. \cite{dalpiaz_runtime_2013} argue that traditional goal models are not enough for reason about the system at runtime. They propose a distinction between Desgin-time Goal Model (DGM) and Runtime Goal Model (RGM). We extend their proposal of RGM with a component model and runtime strategy selection.

\begin{itemize}

\item \textbf{Goal Instance}: an actual instance of an objective for a given data set.

\item \textbf{Believe}: is the model of an actor about itself and the context. Represents what the agent know.

\item \textbf{Strategy Deployment}: occurs after the strategy selection, is the the binding between the strategy, its capabilities dependencies and its environment dependencies.

\end{itemize}

In the proposed model an actor achieve a goal by \emph{deploying a strategy} and executing it. For instance the deployment of a strategy is a capability itself.

\subsection{Strategy Selection and Deployment}

In order to allow component based adaptation we propose a mechanist of strategy selection at runtime. This mechanism is part of the infrastructure layer. For a hight level of flexibility we propose that the selection mechanism itself should be implemented by components in the architecture.

At a low level an agent has the capability of fulfill goals. Inspired by component based frameworks like Rainbow\cite{garlan_rainbow:_2004} we propose that the strategy should be chosen by means of a utility function. That utility function should be responsible to calculate which available strategy will have a better contribution for softgoals.

After a strategy is selected it is deployed. Strategy deployment corresponds to a component binding in CBSE.

The capability <fulfill goal> and its corresponding strategy should be as follows:

\begin{figure}
  \centering
  \includegraphics[width=\linewidth]{strategy_deployment}
  \caption{The deployment of a strategy}
  \label{fig:agent_composition}
\end{figure}

\begin{itemize}
\item \textbf{<Fulfill Goals>} the capability to fulfill generic goals.

\item \textbf{<Fulfill Goals>} strategy to fulfill goals by selecting available strategies and evaluating them with an utility function. Consists of 3 sub-goals:
  \begin{itemize}
    \item Find Matching Strategies
    \item Decide on Strategies
    \item Deploy the Selected Strategy
  \end{itemize}
\end{itemize}

And the following three strategies implement the previous 3 capabilities.

\begin{itemize}
  \item \textbf{<Find Local Matching Strategies>} accomplishes <Find Matching Strategies>
  returns the list of matching strategies. A matching strategy is any strategy that implements the goal interface.

  \item \textbf{<Select a Strategy>} accomplishes <Decide on Strategies>
  using a pre-configured utility function that analyses strategies metadata and select a strategy.

  \item \textbf{<Deploy Strategy>} accomplishes <Deploy the Selected Strategy>. Bind the components with the goal instance and dependencies.
\end{itemize}

\subsection{Awareness}

The agent have a model of that repository that it can use for reason about its capabilities, strategies and plans. The agent is also able to manage its own repository.

Self-awareness is provided by a set of self-awareness strategies. Example of self-aware strategies are:

\begin{itemize}
  \item \textbf{<List Local Strategies>}, how the agent can know it actual capabilities.

  \item \textbf{<List Goal Instances>}, how the agent can know it current intentions.

  \item \textbf{<List Deployed Strategies>}, how the agent can know it current running strategies.

\end{itemize}


\subsection{Multi Agent Deployment}

In order to support multi-agent deployments we will extend the conceptual model and create a set of strategies at the adaptation level. First, in the conceptual model, we propose a extension of the Tropos model concept of \textit{role}.

A role, in runtime, will be associated with a set of capabilities. A role is interpreted also as responsibilities of an actor. So if an agent accepts a role it should keep active strategies to accomplish goals related to capabilities associated with its role.
We will call the state when an actor is satisfying all its assigned roles as \textit{satisfying actor}.

The deployment is the process of assign all needed roles to available agents and moving them to a state of satisfying actors.

In order to allow a system to adapt to deployment variability we propose an \textit{actor verification strategy}. With this strategy the actor should check if itself is currently a satisfying actor. If there is a capability that the agent is currently not satisfying it can use \textit{recovery strategies}.

Recovery strategies could be:
\begin{itemize}
  \item lookup for strategy in external strategy repository;
  \item redistribute roles with peers;
  \item lookup for a peer with the missing capability;
  \item restore resource availability;
  %\item degradate the system, making the capability known as not available
\end{itemize}

\subsection{Virtual Strategies}

To extend the model with adaptation capabilities we propose \textif{Virtual Strategies} (VS). VS are means of achieving a goal that is not by binding a component. We propose two kinds of VS:

\begin{itemize}
  \item \textbf{delegating strategies}. VS that allow an agent to achieve a goal by delegating it to a peer agent. When a agent discover that a peer have some capabilities it creates proxy strategies in its own repository that allow it to fulfill goals by delegating to the peer.
  \item \textbf{fault tolerance strategies}. VS that wrap other strategies in a fault tolerance scheme. Examples of possible fault tolerance strategies are \textit{active replication} and \textit{retry}.
\end{itemize}

The creation of virtual strategies will be performed by \textit{Virtual Strategy Producers} (VSP) that are strategies in itself. The creation of a VSP is a \textit{Virtual Strategy Instance} VSI.

A VSI has its own metadata and is evaluated by the utility function in the process of strategy selection.
%TODO For example, if agent has a goal G1, that needs capability C1 to be accomplish and this capability is implemented by strategies S1, S2. An it has also an \textit{active replication VSP} it will chose between deploying S1, S2 or (S1 || S2). In the example, (S1 || S2) been a VSI, the result of active replication of S1 and S2. With this we expect to allow for the development of portable fault tolerance strategies.

% TODO  Strategies and Virtual Strategies can be combined to make flexible solutions at runtime. For example if an actor has a Goal G1, that needs capability C1 to be accomplish and this capability is implemented by strategies S1, S2, S3.

 A VSP can reason about the context in order to adapt the strategy creation. So, for example, in a context that the agent needs to save battery, the active replication strategy may not be produced.


%\input{content}
%\chapter{Evaluation}
%\label{sec:evaluation}

In this chapter, we focus on the evaluation of the proposed approach.
To do so we used the Goal-Question-Metric (GQM) evaluation methodology~\cite{basili_goal_1994}.

Our first evaluation goal G1 is to assess the feasibility of the approach. To do so, we need to evaluate if a software architect/developer can follow the proposed patterns to refine a goal model into components and artifacts. Also, we need to evaluate if the proposed planning algorithm is capable of autonomously creating a reliable deployment plan.
Such an evaluation required the definition of the following questions and metrics:

\begin{itemize}
  \item Q1.1: For the Filling Station Advisor case study, are the goal-component-artifact patterns a feasible approach to map artifacts from the CGM of the case study?
  \begin{itemize}
    \item Accurately maps artifacts for the Filling Station Advisor case study using proposed patterns.
  \end{itemize}

  \item Q1.2: How long would the algorithm take to come up with a deployment plan?
  \begin{itemize}
    \item Time to produce a plan.
  \end{itemize}

  \item Q1.3: How reliable would a plan provided
  by the algorithm be?
  \begin{itemize}
    \item Percentage of correct answers.
  \end{itemize}

\end{itemize}

Since the Filling Station Advisor has a limited size and does not allow for controlled factors experiments, our second goal G2 aims to provide a more comprehensible scalability evaluation of Goalp. So we defined the following questions and metrics:

\begin{itemize}
  \item Q2.1: How does the algorithm scale over the number of artifacts in the deployment plan?
  \begin{itemize}
    \item M2.1: The time consumed to come up with a deployment plan.
  \end{itemize}

  In the context of heterogeneity, we can have many artifacts in the repository that provide the same goal but with different context conditions.
  We named variability level the number of artifacts present in the repository that provide the same goal. It can affect the scalability of the planning because it leads the algorithm to verify alternative dependency trees, which can be computing intensive.

  \item Q2.2: How does the algorithm scale over the variability level on the repository?
  \begin{itemize}
    \item M2.2: The time consumed to come up with a deployment plan.
  \end{itemize}
\end{itemize}

\subsection{Feasibility Assessment}

We validated the feasibility of the approach applying it to the Filling Station Advisor.

The experiments were conducted using a laptop computer with Intel i5-3337U, 12GB DDR3 1600MHz memory, and Linux (Kernel 3.16.0-77generic). OracleJDK(1.8.0 91-b14) was used to build and run the project.
The experiments to evaluate the algorithm correctness
were implemented as automated tests under Java’s JUnit framework.

The code used to execute the evaluation is available on a public repository
\footnote{The evaluation experiments and step by step guide are available at:
\url{https://github.com/lesunb/goalp} Accessed on December 4th, 2016}
as well as the data obtained and scripts used to treat it.
\footnote{The dataset obtained and the R-Script\cite{the_r_foundation_r_2016} used to analyze the dataset is available at:
\url{https://github.com/lesunb/goalp-evaluation/tree/master/scalability/exp2} Accessed on December 4th, 2016}

\emph{Question 1.1, mapping components and artifacts }

We applied the patterns described in Table~\ref{table_cgm_to_components_patterns} to the CGM depicted in Figure~\ref{fig:goal_model_filling_station_advisor}. Then we defined the artifacts that would package that components following the proposed deployment architecture style (\ref{depl_arch_style}). We then mapped 21 different artifacts.

\emph{Question 1.2 and 1.3}

We instantiated an artifact repository with the mapped artifacts. We defined 7 deployment scenarios under different contexts. The scenarios that we used where: (s1) simple phone with ODB2, (s2) smartphone with ODB2, (s3) smartphone without car connection, (s4) dash computer with GPS and no nav sys integration and (s5) dash computer, connected, with GPS and navigation system integration. Scenarios (s6) dash computer without GPS and (s7) nav system without Internet connection or storage are scenarios for which there is no valid deployment plan.

\begin{figure*}[!htb]
 \centering
 \includegraphics[width=.8\linewidth]{case_study/comp_env_scearios}
 \caption{Computing Environment Evaluation Scenarios}
\label{fig:variability_scenarios}
\end{figure*}

\emph{Question 1.2:  How reliable would a plan provided
by the algorithm be?}: Test cases were created for each scenario (s1-s7).
To validate the algorithm’s correctness,
we verified the generated plans in each test case, asserting if the expected artifacts are in the resulting plan.
For scenarios s1-s5, the planning resulted in valid plans, with the correct artifacts. For scenarios s6 and s7, the algorithm returned \texttt{NULL}, as there is no possible deployment plan for these scenarios.

\begin{figure*}[!htb]
  \centering
  \epsfig{file=results/fsa_testcases, width=3.2in}
  \caption{Passing Tests}
\label{testcase}
\end{figure*}



\emph{Question 1.3: How long would the algorithm take to come up with a deployment plan?}: In each scenario, the time spent by the algorithm was measured. We executed the planning 100 time. Table~\ref{table:planning_time} shows the scenarios, the context, time spent for planning in each scenario, in mile-seconds together with standard deviation.

\begin{table}[!htb]
\centering
\caption{Time to come up with a plan}
\begin{tabular}{|p{0.7cm}|p{3.75cm}|p{2cm}|p{2cm}|}
\hline
  Ref. &
  Context &
  Time (ms) &
  Std \\ \hline

s1 &
C2, C4, C6, C9 &
12.28 ms & 30.69 \\ \hline
s2 &
C1, C3, C5, C8 &
6.24 ms & 16.22 \\ \hline
s3 &
C1, C5, C8 &
9.27 ms & 20.62\\ \hline
s4 &
C1, C3, C6, C10 &
9.01 ms & 20.94 \\ \hline
s5 &
C1, C3, C5, C7 &
6.83 ms & 17.18 \\ \hline
s6 &
C3, C6, C8 &
8.74 ms & 18.76 \\ \hline
s7 &
C1, C3, C7  &
6.44 ms & 17.51 \\ \hline

\end{tabular}
\label{table:planning_time}
\end{table}

    %
    % \item How does it scale over the amount of artifacts in the component repository?
    % \begin{itemize}
    %   \item time to plan the deployment in.
    %   \item space occupied during the planning.
    % \end{itemize}

\subsection{Scalability Assessment}

To evaluate the algorithm's scalability, we developed other test cases.  A repository with randomly generated artifacts was instantiated. And deployment requests that generate plans with a different number of artifacts were made. With this, we could evaluate the impact of the generated plan size in the planning time.
The generated repository had 143,500 artifacts.

The experiments were conducted using a virtual machine in the Azure Cloud. It was used an F1 instance, with 2.4 GHz Intel Xeon® E5-2673 v3 (Haswell) processor,
2GB DDR3 1600MHz memory, and Linux (Kernel 4.4.0-47-generic). OpenJDK(1.9 64bits-build 9) was used to build and run the project.

The code used to execute the evaluation is available on a public repository
\footnote{The needed source code and a step by step guide are available at
\url{https://github.com/lesunb/goalp/tree/master/scalability-evaluation} Accessed on December 4th, 2016}
as well as
the data obtained and scripts used to handle data and plot graphs.
\footnote{R Scripts\cite{the_r_foundation_r_2016} and used dataset are available at
\url{https://github.com/lesunb/goalp/tree/master/scalability-evaluation} Accessed on December 4th, 2016}

\emph{Q2.1: How does the algorithm scale over the number of artifacts in the deployment plan?} We executed 100 deployment planning requests, with different levels of complexity, where the generated plans were composed of artifacts summing from 40 to 3,100 artifacts.

\begin{figure}[!htb]
  \centering
  \epsfig{file=results/planning/plan_size_vs_time.eps, width=5.2in}
  \caption{Scalability over the size of plan}
\label{graph_plan_size_and_time}
\end{figure}

The observed time in function on the number of artifacts in the plan is shown in Figure~\ref{graph_plan_size_and_time}.

\emph{Q2.2:  How does the algorithm scale over the variability level on the repository?}
We repeated the experiment for different levels of variability in the repository, from 1 to 10. A variability level of 1 being so that for each plan implementation there was just one artifact that implement the plan. While in variability level 2, for each plan implementation there was two artifacts, and so on.

The result is depicted in Figure~\ref{graph_scalability}. Each curve represents a different level of variability.

\begin{figure}[!htb]
  \centering
  \epsfig{file=results/planning/size_and_variability.eps, width=5.2in}
  \caption{Scalability over variability level - Average (10 executions)}
\label{graph_scalability}
\end{figure}

At the worst case, a deployment request that needs 3,100 artifacts, with 10 variants for each artifact, took less than 5s to be planned. Requests that required up to 1,000 artifacts could be fulfilled in less then a half-second.

\subsection{Discussion of Results}

In conclusion, the time spent planning the deployment is expected to be negligible in face the time that would take to copy the artifacts from a repository to the target environment.

%\chapter{Conclusion}
%%Conclusions(0.75p)
\section{Conclusion and future work}
\label{sec:conclusion}

% TODO relate with multi-level adaptation.
In this work, we presented Goalp, a novel approach to tackle deployment in highly heterogeneous computing environments.
Goalp allows systems deployment to heterogeneous environments, partially unknown at design-time, without requiring a system administrator.
Goalp consists in support to design a system with the needed variability to handle the heterogeneity, from requirements, through architecture, and deployment.
And in online support for solving the variability at deployment time, finding the correct set of artifacts that allows the user achieve its goals in a given target computing environment. Goalp uses a CGM to specify variability at requirements. Further, patterns are used to map components from the CGM and keep the variability are architecture level and deployment level. The novelty of our approach is that we provide a systematic way to design a system with focus in variability from requirements to deployment.
% As such, using a goal-oriented approach to deployment is expected to integrate with other approach that handle variability in another level.

Following our approach the system implemented reflects the goal-model, keeping the goals traceable to components and artifacts. Via such traceability the adequate set of artifacts is autonomously chosen achieving the target software goal in a given computing environment. Since goal models are highly abstract models, using it to drive the system adaptation, we expect to achieve a higher level of flexibility transcending the lower-level abstraction computing layers. In addition, by using context-goal models, we can handle computing resources variability. By using CGM for deployment, rework is avoided, as CGM is a model already developed in the requirements elicitation stage.

In a preliminary evaluation, we applied the Goalp approach in a case study. Further, we evaluated the scalability of the algorithm when planning in a large scenario, using a randomly generated repository and deployment requests. The results show that the algorithm is capable of coming up with a plan, in a reasonably large scenario in few seconds.

This work fits in our long-term vision of a method for design systems with variability at all stages of system design, from requirements to deployment. And a self-adaptable platform that can adapt the software deployment in order to make high-level user goals achievable. This work fits in this vision by providing the knowledge and planning part in a MAPE-K\cite{kephart_vision_2003} architecture.
We should note that this work aims at identifying a single valid deployment plan, as long as it exists. However, it is out of scope of our current work to find the best valid plan in case multiple valid plans exist. In future work, a CSP approach might integrate our deployment plan algorithm to address such issue.
For future work, we plan to: (1) extend Goalp with deployment planning for multiple nodes by including delegation as another form of variability;  (2) evolve Goalp deployment planning in a self-adaptive approach for deployment, based on MAPE-K, with addition of monitoring, analyzing, and executing capabilities; (3) evaluate Goalp in an open adaptation scenario with multiple developers providing components to the environment; and (4) evaluate self-adaptation at deployment level as a method of fault-tolerance that adapts the system deployment in response to failures in resources.
%We provide a approach that To the best of our knowledge no other approach has tackle this problem.
%We believe that this is will be useful in domain such as ubiquitous at where no specialized people interact with the system and would like it to adapt its deployment.

%The advantages of goal-driven is:
%High level of flexbility by using the system goals as a model for evaluate alternatives of adaptation.
%Easy the development by reuse the goal model.
%The advantages of using a MAS approach is the distributed nature of the system and to avoid have a sigle point of failure.


\postextual
\anexos


\bibliographystyle{plain}
\bibliography{bibliografia}
\end{document}
