\section{Conceptual Model}
\label{conceptual_model}

\subsection{Component Model}

We propose and extension to Tropos Model with a component model.By this we aim at creating an appropriate abstraction to allow composable architecture while keeping the traceability between the requirements and implementation at runtime.

Or component model is build around the concept of \textit{strategy}. The strategy has a double meaning. It is an alternative way of achieving a goal (as in GORE). And also it is a component in the architecture. A capacity is a interface description the component model.
From a goal model, each goal will originate a capacity and strategy. Each OR decomposition of a goal, in the goal model, should correspond to an additional strategy.
An agent in the system has a repository of strategies.

The main concepts are:

\begin{itemize}
  \item \textbf{Agent}: agent is an independent computational unit that manages its own resources (CPU, memory, disk, sensors, etc). After system deployment an agent is turned into an actor of a Tropos model.
  \item \textbf{Capability}: description of a kind of goal that an agent can perform. Its a component interface description in the architecture. (e.g SUM a and b)
  \item \textbf{Strategy}: a strategy is a capability alternative implementation (e.g (SUM,a,b) => {a+b}). Its also a module in the architecture. A strategy should implement one capability.
  \item \textbf{Strategy Repository}: an agent has a repository of its capabilities and strategies.
  \item \textbf{System Deployment}: the process of putting the right strategies in agents repositories.
\end{itemize}

\begin{figure}
  \centering
  \includegraphics[width=\linewidth]{goalp-agent-repo-rcm-depl}
  \caption{Agent}
  \label{fig:goalp-agent}
\end{figure}

\subsection{Architectural Layers}

For the implementation of the conceptual model, we propose an architecture of three layers:

\begin{itemize}
  \item \textbf{Infrastructure}: This layer is responsible for the component model implementation, how capability, strategies, plans and goals instances are described. It also responsible for agent and goals life cycle, strategies management and selection.
  \item \textbf{Adaptation}: This layer is responsible for keep the level of service. It contains strategies to discover peers, team up with peers, gather information to improve strategy selection, etc.
  \item \textbf{Application}: This layer if responsible for functional strategies. The functional strategies are the one that implement user application.
\end{itemize}

\subsection{Strategy Deployment}

In order to allow component based adaptation we propose a mechanist of strategy selection at runtime. This mechanism is part of the infrastructure layer. For a hight level of flexibility we propose that the selection mechanism itself should be implemented by components in the architecture.

At a low level an agent has the capability of fulfill goals. Inspired by component based frameworks like Rainbow\cite{garlan_rainbow:_2004} we propose that the strategy should be chosen by means of a utility function. That utility function should be responsible to calculate which available strategy will have a better contribution for softgoals.

The capacity and strategy should be as follows:

\begin{itemize}
\item \textbf{<Fulfill Goals> Goal} the capability of fulfill generic goals.

\item \textbf{<Fulfill Goals> Strategy } strategy to fulfill goals by selecting available strategies and evaluating them with an utility function. Consist of 3 sub-goals:
  \begin{itemize}
    \item Find Matching Strategies
    \item Decide on Strategies
    \item Deploy the Selected Strategy
  \end{itemize}
\end{itemize}

And the following three strategies implement the previous 3 capabilities.

\begin{itemize}
  \item \textbf{<Find Local Matching Strategies> Strategy} accomplish <Find Matching Strategies>
  return the list of matching strategies. A matching strategy is any strategy that implement the goal interface.

  \item \textbf{<Select a Strategy> Strategy} accomplish <Decide on Strategies>
  use a pre-configured utility function that analyses strategies metadata and select a strategy.

  \item \textbf{<Deploy Strategy> Strategy} accomplish <Deploy the Selected Strategy>.
  Consist in deploying the chosen component.
\end{itemize}

\begin{figure}
  \centering
  \includegraphics[width=\linewidth]{strategy_deployment}
  \caption{The deployment of a strategy}
  \label{fig:agent_composition}
\end{figure}

\subsection{Runtime Model}

Dalpiaz et al. \cite{dalpiaz_runtime_2013} argue that traditional goal models are not enough for reason about the system at runtime. They propose a distinction between Desgin-time Goal Model and Runtime Goal Model. We extend their proposal with model at runtime of the agent.

\begin{itemize}

\item \textbf{Goal Instance}: an actual instance of an objective for a given data set. (e.g SUM 2 and 3)

\item \textbf{Believe}: in the knowledge of an actor about itself and the context.

\item \textbf{Strategy Deployment}: occurs after the strategy selection, is the the binding between the strategy, its dependencies and its environment dependencies.

\end{itemize}

In the proposed model an actor achieve a goal by \emph{deploying a strategy}. For instance the deployment of a strategy is a capability itself, as follows:



\subsection{Awareness}

The agent have a model of that repository that it can use for reason about its capacities, strategies and plans. The agent is also able to manage its own repository.

Self-awareness is provided by a set of self-awareness strategies. Example of self-aware strategies are:

\begin{itemize}
  \item \textbf{<List Local Strategies> Strategy} How the agent can know it actual capabilities.

  \item \textbf{<List Goal Instances> Strategy} How the agent can know it current intentions.

  \item \textbf{<List Deployed Strategies> Strategy} How the agent can know it current running strategies.

\end{itemize}


%
% \section{Resources}
%
% \section{Strategy Declaration}
%
%
% \section{Fault Model}
% Consistent/Inconsistent Failure
%
% Agent failure
%
% Resource failure
%
% Strategy failure
